% Options for packages loaded elsewhere
\PassOptionsToPackage{unicode}{hyperref}
\PassOptionsToPackage{hyphens}{url}
%
\documentclass[
]{article}
\usepackage{amsmath,amssymb}
\usepackage{iftex}
\ifPDFTeX
  \usepackage[T1]{fontenc}
  \usepackage[utf8]{inputenc}
  \usepackage{textcomp} % provide euro and other symbols
\else % if luatex or xetex
  \usepackage{unicode-math} % this also loads fontspec
  \defaultfontfeatures{Scale=MatchLowercase}
  \defaultfontfeatures[\rmfamily]{Ligatures=TeX,Scale=1}
\fi
\usepackage{lmodern}
\ifPDFTeX\else
  % xetex/luatex font selection
\fi
% Use upquote if available, for straight quotes in verbatim environments
\IfFileExists{upquote.sty}{\usepackage{upquote}}{}
\IfFileExists{microtype.sty}{% use microtype if available
  \usepackage[]{microtype}
  \UseMicrotypeSet[protrusion]{basicmath} % disable protrusion for tt fonts
}{}
\makeatletter
\@ifundefined{KOMAClassName}{% if non-KOMA class
  \IfFileExists{parskip.sty}{%
    \usepackage{parskip}
  }{% else
    \setlength{\parindent}{0pt}
    \setlength{\parskip}{6pt plus 2pt minus 1pt}}
}{% if KOMA class
  \KOMAoptions{parskip=half}}
\makeatother
\usepackage{xcolor}
\usepackage[margin=1in]{geometry}
\usepackage{color}
\usepackage{fancyvrb}
\newcommand{\VerbBar}{|}
\newcommand{\VERB}{\Verb[commandchars=\\\{\}]}
\DefineVerbatimEnvironment{Highlighting}{Verbatim}{commandchars=\\\{\}}
% Add ',fontsize=\small' for more characters per line
\usepackage{framed}
\definecolor{shadecolor}{RGB}{248,248,248}
\newenvironment{Shaded}{\begin{snugshade}}{\end{snugshade}}
\newcommand{\AlertTok}[1]{\textcolor[rgb]{0.94,0.16,0.16}{#1}}
\newcommand{\AnnotationTok}[1]{\textcolor[rgb]{0.56,0.35,0.01}{\textbf{\textit{#1}}}}
\newcommand{\AttributeTok}[1]{\textcolor[rgb]{0.13,0.29,0.53}{#1}}
\newcommand{\BaseNTok}[1]{\textcolor[rgb]{0.00,0.00,0.81}{#1}}
\newcommand{\BuiltInTok}[1]{#1}
\newcommand{\CharTok}[1]{\textcolor[rgb]{0.31,0.60,0.02}{#1}}
\newcommand{\CommentTok}[1]{\textcolor[rgb]{0.56,0.35,0.01}{\textit{#1}}}
\newcommand{\CommentVarTok}[1]{\textcolor[rgb]{0.56,0.35,0.01}{\textbf{\textit{#1}}}}
\newcommand{\ConstantTok}[1]{\textcolor[rgb]{0.56,0.35,0.01}{#1}}
\newcommand{\ControlFlowTok}[1]{\textcolor[rgb]{0.13,0.29,0.53}{\textbf{#1}}}
\newcommand{\DataTypeTok}[1]{\textcolor[rgb]{0.13,0.29,0.53}{#1}}
\newcommand{\DecValTok}[1]{\textcolor[rgb]{0.00,0.00,0.81}{#1}}
\newcommand{\DocumentationTok}[1]{\textcolor[rgb]{0.56,0.35,0.01}{\textbf{\textit{#1}}}}
\newcommand{\ErrorTok}[1]{\textcolor[rgb]{0.64,0.00,0.00}{\textbf{#1}}}
\newcommand{\ExtensionTok}[1]{#1}
\newcommand{\FloatTok}[1]{\textcolor[rgb]{0.00,0.00,0.81}{#1}}
\newcommand{\FunctionTok}[1]{\textcolor[rgb]{0.13,0.29,0.53}{\textbf{#1}}}
\newcommand{\ImportTok}[1]{#1}
\newcommand{\InformationTok}[1]{\textcolor[rgb]{0.56,0.35,0.01}{\textbf{\textit{#1}}}}
\newcommand{\KeywordTok}[1]{\textcolor[rgb]{0.13,0.29,0.53}{\textbf{#1}}}
\newcommand{\NormalTok}[1]{#1}
\newcommand{\OperatorTok}[1]{\textcolor[rgb]{0.81,0.36,0.00}{\textbf{#1}}}
\newcommand{\OtherTok}[1]{\textcolor[rgb]{0.56,0.35,0.01}{#1}}
\newcommand{\PreprocessorTok}[1]{\textcolor[rgb]{0.56,0.35,0.01}{\textit{#1}}}
\newcommand{\RegionMarkerTok}[1]{#1}
\newcommand{\SpecialCharTok}[1]{\textcolor[rgb]{0.81,0.36,0.00}{\textbf{#1}}}
\newcommand{\SpecialStringTok}[1]{\textcolor[rgb]{0.31,0.60,0.02}{#1}}
\newcommand{\StringTok}[1]{\textcolor[rgb]{0.31,0.60,0.02}{#1}}
\newcommand{\VariableTok}[1]{\textcolor[rgb]{0.00,0.00,0.00}{#1}}
\newcommand{\VerbatimStringTok}[1]{\textcolor[rgb]{0.31,0.60,0.02}{#1}}
\newcommand{\WarningTok}[1]{\textcolor[rgb]{0.56,0.35,0.01}{\textbf{\textit{#1}}}}
\usepackage{longtable,booktabs,array}
\usepackage{calc} % for calculating minipage widths
% Correct order of tables after \paragraph or \subparagraph
\usepackage{etoolbox}
\makeatletter
\patchcmd\longtable{\par}{\if@noskipsec\mbox{}\fi\par}{}{}
\makeatother
% Allow footnotes in longtable head/foot
\IfFileExists{footnotehyper.sty}{\usepackage{footnotehyper}}{\usepackage{footnote}}
\makesavenoteenv{longtable}
\usepackage{graphicx}
\makeatletter
\def\maxwidth{\ifdim\Gin@nat@width>\linewidth\linewidth\else\Gin@nat@width\fi}
\def\maxheight{\ifdim\Gin@nat@height>\textheight\textheight\else\Gin@nat@height\fi}
\makeatother
% Scale images if necessary, so that they will not overflow the page
% margins by default, and it is still possible to overwrite the defaults
% using explicit options in \includegraphics[width, height, ...]{}
\setkeys{Gin}{width=\maxwidth,height=\maxheight,keepaspectratio}
% Set default figure placement to htbp
\makeatletter
\def\fps@figure{htbp}
\makeatother
\setlength{\emergencystretch}{3em} % prevent overfull lines
\providecommand{\tightlist}{%
  \setlength{\itemsep}{0pt}\setlength{\parskip}{0pt}}
\setcounter{secnumdepth}{5}
\ifLuaTeX
\usepackage[bidi=basic]{babel}
\else
\usepackage[bidi=default]{babel}
\fi
\babelprovide[main,import]{spanish}
% get rid of language-specific shorthands (see #6817):
\let\LanguageShortHands\languageshorthands
\def\languageshorthands#1{}
\usepackage{booktabs}
\usepackage{longtable}
\usepackage{array}
\usepackage{multirow}
\usepackage{wrapfig}
\usepackage{float}
\usepackage{colortbl}
\usepackage{pdflscape}
\usepackage{tabu}
\usepackage{threeparttable}
\usepackage{threeparttablex}
\usepackage[normalem]{ulem}
\usepackage{makecell}
\usepackage{xcolor}
\ifLuaTeX
  \usepackage{selnolig}  % disable illegal ligatures
\fi
\usepackage{bookmark}
\IfFileExists{xurl.sty}{\usepackage{xurl}}{} % add URL line breaks if available
\urlstyle{same}
\hypersetup{
  pdftitle={Resolución Problemas},
  pdflang={es},
  hidelinks,
  pdfcreator={LaTeX via pandoc}}

\title{Resolución Problemas}
\author{}
\date{\vspace{-2.5em}}

\begin{document}
\maketitle

{
\setcounter{tocdepth}{5}
\tableofcontents
}
\newpage

\section{Laura Mayorgasdel Castillo}\label{laura-mayorgasdel-castillo}

\subsection{Problema 1}\label{problema-1}

Aplicar los criterios de decisión bajo incertidumbre a los problemas
cuya matriz de valores numéricos vienen dadas en la tabla siguiente:

\begin{longtable}[]{@{}lrrrr@{}}
\toprule\noalign{}
& e1 & e2 & e3 & e4 \\
\midrule\noalign{}
\endhead
\bottomrule\noalign{}
\endlastfoot
d1 & 4 & 3 & -4 & 4 \\
d2 & 5 & -1 & 9 & 6 \\
d3 & -3 & 3 & 8 & 7 \\
d4 & 7 & 7 & 2 & -9 \\
d5 & 8 & 9 & 1 & 3 \\
\end{longtable}

\begin{enumerate}
\def\labelenumi{\alph{enumi})}
\tightlist
\item
  Desde el punto de vista FAVORABLE
\item
  Desde el punto de vista DESFAVORABLE
\end{enumerate}

\subsubsection{Desde el punto de vista FAVORABLE o
BENEFICIOS}\label{desde-el-punto-de-vista-favorable-o-beneficios}

\begin{Shaded}
\begin{Highlighting}[]
\NormalTok{tb01}\OtherTok{=} \FunctionTok{crea.tablaX}\NormalTok{(}\FunctionTok{c}\NormalTok{(}\DecValTok{4}\NormalTok{,}\DecValTok{5}\NormalTok{,}\SpecialCharTok{{-}}\DecValTok{3}\NormalTok{,}\DecValTok{7}\NormalTok{,}\DecValTok{8}\NormalTok{,}
                    \DecValTok{3}\NormalTok{,}\SpecialCharTok{{-}}\DecValTok{1}\NormalTok{,}\DecValTok{3}\NormalTok{,}\DecValTok{7}\NormalTok{,}\DecValTok{9}\NormalTok{,}
                    \SpecialCharTok{{-}}\DecValTok{4}\NormalTok{,}\DecValTok{9}\NormalTok{,}\DecValTok{8}\NormalTok{,}\DecValTok{2}\NormalTok{,}\DecValTok{1}\NormalTok{,}
                    \DecValTok{4}\NormalTok{,}\DecValTok{6}\NormalTok{,}\DecValTok{7}\NormalTok{,}\SpecialCharTok{{-}}\DecValTok{9}\NormalTok{,}\DecValTok{3}\NormalTok{),}\AttributeTok{numalternativas =} \DecValTok{4}\NormalTok{,}\AttributeTok{numestados =} \DecValTok{5}\NormalTok{)}
\end{Highlighting}
\end{Shaded}

\begin{enumerate}
\def\labelenumi{\arabic{enumi}.}
\tightlist
\item
  Criterio de Wald
\end{enumerate}

\begin{Shaded}
\begin{Highlighting}[]
\NormalTok{sol1Wlad}\OtherTok{=}\FunctionTok{criterio.Wald}\NormalTok{(tb01,}\AttributeTok{favorable =} \ConstantTok{TRUE}\NormalTok{)}
\FunctionTok{cat}\NormalTok{(}\StringTok{"La solución por el criterio de Wald es la:"}\NormalTok{, sol1Wlad}\SpecialCharTok{$}\NormalTok{AlternativaOptima)}
\end{Highlighting}
\end{Shaded}

\begin{verbatim}
## La solución por el criterio de Wald es la: 2
\end{verbatim}

\begin{enumerate}
\def\labelenumi{\arabic{enumi}.}
\setcounter{enumi}{1}
\tightlist
\item
  Criterio optimista
\end{enumerate}

\begin{Shaded}
\begin{Highlighting}[]
\NormalTok{sol1Opt}\OtherTok{=}\FunctionTok{criterio.Optimista}\NormalTok{(tb01,}\AttributeTok{favorable =} \ConstantTok{TRUE}\NormalTok{)}
\FunctionTok{cat}\NormalTok{(}\StringTok{"La solución por el criterio Optimista es la:"}\NormalTok{, sol1Opt}\SpecialCharTok{$}\NormalTok{AlternativaOptima)}
\end{Highlighting}
\end{Shaded}

\begin{verbatim}
## La solución por el criterio Optimista es la: 2 3
\end{verbatim}

\begin{enumerate}
\def\labelenumi{\arabic{enumi}.}
\setcounter{enumi}{2}
\tightlist
\item
  Criterio de Hurwitcz
\end{enumerate}

\begin{Shaded}
\begin{Highlighting}[]
\NormalTok{sol1Hur}\OtherTok{=}\FunctionTok{criterio.Hurwicz}\NormalTok{(tb01,}\AttributeTok{favorable =} \ConstantTok{TRUE}\NormalTok{)}
\FunctionTok{cat}\NormalTok{(}\StringTok{"La solución por el criterio de Hurwitcz es la:"}\NormalTok{, sol1Hur}\SpecialCharTok{$}\NormalTok{AlternativaOptima)}
\end{Highlighting}
\end{Shaded}

\begin{verbatim}
## La solución por el criterio de Hurwitcz es la: 2
\end{verbatim}

\begin{Shaded}
\begin{Highlighting}[]
\FunctionTok{dibuja.criterio.Hurwicz}\NormalTok{(tb01,}\AttributeTok{favorable =} \ConstantTok{TRUE}\NormalTok{)}
\end{Highlighting}
\end{Shaded}

\includegraphics{Resolucion_files/figure-latex/unnamed-chunk-6-1.pdf}

\begin{enumerate}
\def\labelenumi{\arabic{enumi}.}
\setcounter{enumi}{3}
\tightlist
\item
  Criterio de Savage
\end{enumerate}

\begin{Shaded}
\begin{Highlighting}[]
\NormalTok{sol1Sav}\OtherTok{=}\FunctionTok{criterio.Savage}\NormalTok{(tb01,}\AttributeTok{favorable =} \ConstantTok{TRUE}\NormalTok{)}
\FunctionTok{cat}\NormalTok{(}\StringTok{"La solución por el criterio de Savage es la:"}\NormalTok{, sol1Sav}\SpecialCharTok{$}\NormalTok{AlternativaOptima)}
\end{Highlighting}
\end{Shaded}

\begin{verbatim}
## La solución por el criterio de Savage es la: 3
\end{verbatim}

\begin{enumerate}
\def\labelenumi{\arabic{enumi}.}
\setcounter{enumi}{4}
\tightlist
\item
  Criterio de Lapalce
\end{enumerate}

\begin{Shaded}
\begin{Highlighting}[]
\NormalTok{sol1Lap}\OtherTok{=}\FunctionTok{criterio.Laplace}\NormalTok{(tb01,}\AttributeTok{favorable =} \ConstantTok{TRUE}\NormalTok{)}
\FunctionTok{cat}\NormalTok{(}\StringTok{"La solución por el criterio Laplace es la:"}\NormalTok{, sol1Lap}\SpecialCharTok{$}\NormalTok{AlternativaOptima)}
\end{Highlighting}
\end{Shaded}

\begin{verbatim}
## La solución por el criterio Laplace es la: 1 2
\end{verbatim}

\begin{enumerate}
\def\labelenumi{\arabic{enumi}.}
\setcounter{enumi}{6}
\tightlist
\item
  Criterio del punto ideal
\end{enumerate}

\begin{Shaded}
\begin{Highlighting}[]
\NormalTok{sol1PuntId}\OtherTok{=}\FunctionTok{criterio.PuntoIdeal}\NormalTok{(tb01,}\AttributeTok{favorable =} \ConstantTok{TRUE}\NormalTok{)}
\FunctionTok{cat}\NormalTok{(}\StringTok{"La solución por el criterio del Punto Ideal es la:"}\NormalTok{, sol1PuntId}\SpecialCharTok{$}\NormalTok{AlternativaOptima)}
\end{Highlighting}
\end{Shaded}

\begin{verbatim}
## La solución por el criterio del Punto Ideal es la: 2
\end{verbatim}

\begin{enumerate}
\def\labelenumi{\arabic{enumi}.}
\setcounter{enumi}{7}
\tightlist
\item
  Criterios todos
\end{enumerate}

\begin{Shaded}
\begin{Highlighting}[]
\NormalTok{sol1Todos}\OtherTok{=}\FunctionTok{criterio.Todos}\NormalTok{(tb01,}\AttributeTok{favorable =}\NormalTok{ T, }\AttributeTok{alfa =} \FloatTok{0.5}\NormalTok{)}
\NormalTok{knitr}\SpecialCharTok{::}\FunctionTok{kable}\NormalTok{(sol1Todos[,}\DecValTok{6}\SpecialCharTok{:}\FunctionTok{ncol}\NormalTok{(sol1Todos)])}
\end{Highlighting}
\end{Shaded}

\begin{longtable}[]{@{}
  >{\raggedright\arraybackslash}p{(\columnwidth - 12\tabcolsep) * \real{0.2424}}
  >{\raggedright\arraybackslash}p{(\columnwidth - 12\tabcolsep) * \real{0.0758}}
  >{\raggedright\arraybackslash}p{(\columnwidth - 12\tabcolsep) * \real{0.1515}}
  >{\raggedright\arraybackslash}p{(\columnwidth - 12\tabcolsep) * \real{0.1212}}
  >{\raggedright\arraybackslash}p{(\columnwidth - 12\tabcolsep) * \real{0.1061}}
  >{\raggedright\arraybackslash}p{(\columnwidth - 12\tabcolsep) * \real{0.1212}}
  >{\raggedright\arraybackslash}p{(\columnwidth - 12\tabcolsep) * \real{0.1818}}@{}}
\toprule\noalign{}
\begin{minipage}[b]{\linewidth}\raggedright
\end{minipage} & \begin{minipage}[b]{\linewidth}\raggedright
Wald
\end{minipage} & \begin{minipage}[b]{\linewidth}\raggedright
Optimista
\end{minipage} & \begin{minipage}[b]{\linewidth}\raggedright
Hurwicz
\end{minipage} & \begin{minipage}[b]{\linewidth}\raggedright
Savage
\end{minipage} & \begin{minipage}[b]{\linewidth}\raggedright
Laplace
\end{minipage} & \begin{minipage}[b]{\linewidth}\raggedright
Punto Ideal
\end{minipage} \\
\midrule\noalign{}
\endhead
\bottomrule\noalign{}
\endlastfoot
d1 & -3 & 8 & 2.5 & 11 & 4.2 & 11.75 \\
d2 & -1 & 9 & 4.0 & 10 & 4.2 & 11.22 \\
d3 & -4 & 9 & 2.5 & 8 & 3.2 & 12.37 \\
d4 & -9 & 7 & -1.0 & 16 & 2.2 & 17.38 \\
iAlt.Opt (fav.) & d2 & d2,d3 & d2 & d3 & d1,d2 & d2 \\
\end{longtable}

\subsubsection{Desde el punto de vista DESFAVORABLE o de
COSTES}\label{desde-el-punto-de-vista-desfavorable-o-de-costes}

\begin{enumerate}
\def\labelenumi{\arabic{enumi}.}
\tightlist
\item
  Criterio de Wald
\end{enumerate}

\begin{Shaded}
\begin{Highlighting}[]
\NormalTok{sol2Wlad}\OtherTok{=}\FunctionTok{criterio.Wald}\NormalTok{(tb01,}\AttributeTok{favorable =} \ConstantTok{FALSE}\NormalTok{)}
\FunctionTok{cat}\NormalTok{(}\StringTok{"La solución por el criterio de Wald es la:"}\NormalTok{, sol2Wlad}\SpecialCharTok{$}\NormalTok{AlternativaOptima)}
\end{Highlighting}
\end{Shaded}

\begin{verbatim}
## La solución por el criterio de Wald es la: 4
\end{verbatim}

\begin{enumerate}
\def\labelenumi{\arabic{enumi}.}
\setcounter{enumi}{1}
\tightlist
\item
  Criterio optimista
\end{enumerate}

\begin{Shaded}
\begin{Highlighting}[]
\NormalTok{sol2Opt}\OtherTok{=}\FunctionTok{criterio.Optimista}\NormalTok{(tb01,}\AttributeTok{favorable =} \ConstantTok{FALSE}\NormalTok{)}
\FunctionTok{cat}\NormalTok{(}\StringTok{"La solución por el criterio Optimista es la:"}\NormalTok{, sol2Opt}\SpecialCharTok{$}\NormalTok{AlternativaOptima)}
\end{Highlighting}
\end{Shaded}

\begin{verbatim}
## La solución por el criterio Optimista es la: 4
\end{verbatim}

\newpage
3

. Criterio de Hurwitcz

\begin{Shaded}
\begin{Highlighting}[]
\NormalTok{sol2Hur1}\OtherTok{=}\FunctionTok{criterio.Hurwicz}\NormalTok{(tb01,}\AttributeTok{favorable =} \ConstantTok{FALSE}\NormalTok{,}\AttributeTok{alfa =} \FloatTok{0.4}\NormalTok{)}
\FunctionTok{cat}\NormalTok{(}\StringTok{"La solución por el criterio de Hurwitcz es la:"}\NormalTok{, sol2Hur1}\SpecialCharTok{$}\NormalTok{AlternativaOptima)}
\end{Highlighting}
\end{Shaded}

\begin{verbatim}
## La solución por el criterio de Hurwitcz es la: 4
\end{verbatim}

\begin{Shaded}
\begin{Highlighting}[]
\NormalTok{sol2Hur2}\OtherTok{=} \FunctionTok{criterio.Hurwicz.General}\NormalTok{(tb01,}\AttributeTok{favorable =} \ConstantTok{FALSE}\NormalTok{, }\AttributeTok{alfa =} \FloatTok{0.4}\NormalTok{)}
\FunctionTok{dibuja.criterio.Hurwicz}\NormalTok{(tb01,}\AttributeTok{favorable =} \ConstantTok{FALSE}\NormalTok{)}
\end{Highlighting}
\end{Shaded}

\includegraphics{Resolucion_files/figure-latex/unnamed-chunk-14-1.pdf}

\begin{enumerate}
\def\labelenumi{\arabic{enumi}.}
\setcounter{enumi}{3}
\tightlist
\item
  Criterio de Savage
\end{enumerate}

\begin{Shaded}
\begin{Highlighting}[]
\NormalTok{sol2Sav}\OtherTok{=}\FunctionTok{criterio.Savage}\NormalTok{(tb01,}\AttributeTok{favorable =} \ConstantTok{FALSE}\NormalTok{)}
\FunctionTok{cat}\NormalTok{(}\StringTok{"La solución por el criterio de Savage es la:"}\NormalTok{, sol2Sav}\SpecialCharTok{$}\NormalTok{AlternativaOptima)}
\end{Highlighting}
\end{Shaded}

\begin{verbatim}
## La solución por el criterio de Savage es la: 4
\end{verbatim}

\begin{enumerate}
\def\labelenumi{\arabic{enumi}.}
\setcounter{enumi}{4}
\tightlist
\item
  Criterio de Lapalce
\end{enumerate}

\begin{Shaded}
\begin{Highlighting}[]
\NormalTok{sol2Lap}\OtherTok{=}\FunctionTok{criterio.Laplace}\NormalTok{(tb01,}\AttributeTok{favorable =} \ConstantTok{FALSE}\NormalTok{)}
\FunctionTok{cat}\NormalTok{(}\StringTok{"La solución por el criterio Laplace es la:"}\NormalTok{, sol2Lap}\SpecialCharTok{$}\NormalTok{AlternativaOptima)}
\end{Highlighting}
\end{Shaded}

\begin{verbatim}
## La solución por el criterio Laplace es la: 4
\end{verbatim}

\begin{enumerate}
\def\labelenumi{\arabic{enumi}.}
\setcounter{enumi}{5}
\tightlist
\item
  Criterio del punto ideal
\end{enumerate}

\begin{Shaded}
\begin{Highlighting}[]
\NormalTok{sol2PuntId}\OtherTok{=}\FunctionTok{criterio.PuntoIdeal}\NormalTok{(tb01,}\AttributeTok{favorable =} \ConstantTok{FALSE}\NormalTok{)}
\FunctionTok{cat}\NormalTok{(}\StringTok{"La solución por el criterio del Punto Ideal es la:"}\NormalTok{, sol2PuntId}\SpecialCharTok{$}\NormalTok{AlternativaOptima)}
\end{Highlighting}
\end{Shaded}

\begin{verbatim}
## La solución por el criterio del Punto Ideal es la: 4
\end{verbatim}

\begin{enumerate}
\def\labelenumi{\arabic{enumi}.}
\setcounter{enumi}{6}
\tightlist
\item
  Criterios todos
\end{enumerate}

\begin{Shaded}
\begin{Highlighting}[]
\NormalTok{sol2Todos}\OtherTok{=}\FunctionTok{criterio.Todos}\NormalTok{(tb01,}\AttributeTok{favorable =} \ConstantTok{FALSE}\NormalTok{, }\AttributeTok{alfa =} \FloatTok{0.5}\NormalTok{)}
\NormalTok{knitr}\SpecialCharTok{::}\FunctionTok{kable}\NormalTok{(sol2Todos[,}\DecValTok{6}\SpecialCharTok{:}\FunctionTok{ncol}\NormalTok{(sol2Todos)])}
\end{Highlighting}
\end{Shaded}

\begin{longtable}[]{@{}
  >{\raggedright\arraybackslash}p{(\columnwidth - 12\tabcolsep) * \real{0.2754}}
  >{\raggedright\arraybackslash}p{(\columnwidth - 12\tabcolsep) * \real{0.0725}}
  >{\raggedright\arraybackslash}p{(\columnwidth - 12\tabcolsep) * \real{0.1449}}
  >{\raggedright\arraybackslash}p{(\columnwidth - 12\tabcolsep) * \real{0.1159}}
  >{\raggedright\arraybackslash}p{(\columnwidth - 12\tabcolsep) * \real{0.1014}}
  >{\raggedright\arraybackslash}p{(\columnwidth - 12\tabcolsep) * \real{0.1159}}
  >{\raggedright\arraybackslash}p{(\columnwidth - 12\tabcolsep) * \real{0.1739}}@{}}
\toprule\noalign{}
\begin{minipage}[b]{\linewidth}\raggedright
\end{minipage} & \begin{minipage}[b]{\linewidth}\raggedright
Wald
\end{minipage} & \begin{minipage}[b]{\linewidth}\raggedright
Optimista
\end{minipage} & \begin{minipage}[b]{\linewidth}\raggedright
Hurwicz
\end{minipage} & \begin{minipage}[b]{\linewidth}\raggedright
Savage
\end{minipage} & \begin{minipage}[b]{\linewidth}\raggedright
Laplace
\end{minipage} & \begin{minipage}[b]{\linewidth}\raggedright
Punto Ideal
\end{minipage} \\
\midrule\noalign{}
\endhead
\bottomrule\noalign{}
\endlastfoot
d1 & 8 & -3 & 2.5 & 16 & 4.2 & 20.12 \\
d2 & 9 & -1 & 4.0 & 16 & 4.2 & 20.12 \\
d3 & 9 & -4 & 2.5 & 11 & 3.2 & 18.49 \\
d4 & 7 & -9 & -1.0 & 10 & 2.2 & 14.73 \\
iAlt.Opt (Desfav.) & d4 & d4 & d4 & d4 & d4 & d4 \\
\end{longtable}

\newpage

\subsection{Problema 2}\label{problema-2}

El equipo ARUS de Formula Student de la Universidad de Sevilla está
evaluando a qué competición asistir este año. Actualmente, tienen tres
opciones disponibles, y cada una tiene características distintas en
términos de coste, nivel de competencia y beneficios potenciales: Para
acceder a cada una se debe pagar una cuota de inscripción

\textbf{Competición en Barcelona:} La cuota de inscripción es de 2,000
euros y el coste de transporte es bajo ya que no saldrían de España,
aproximadamente de unos 1,000 euros. El equipo espera obtener buenos
resultados aquí ya que se encuentran en casa, en caso de ganar, podrían
conseguir un premio de 5,000 euros. Sin embargo, el nivel de competencia
es medio bajo teniendo muchas posibilidades de quedar entre los
primeros.

\textbf{Competición en Hockenheim (Alemania):} Es una de la competición
más prestigiosa, esto se ve reflejado en su cuota de inscripción que es
de 3,500 euros. Además el coste de transporte debido a las escasas
conexiones es elevado llegando a 4,000 euros. Si logran quedar entre los
primeros puestos, podrían conseguir un premio de 10,000 euros.Una
increible recompensa que sumada al orgullo de ganarla es un gran
aliciente.

\textbf{Competición en Assen (Paises Bajos):} El coste de inscripción es
intermedio, 2,800 euros, y el coste de transporte es de 3,500 euros. Es
una competición con buen reconocimiento pero que no lleva tantos años
celebrandose. El premio es una buena cifra tratándose de 7,000 euros, y
el nivel de competencia también es alto.

¿A qué competición le recomendarías a la directiva del equipo asistir?

Costes y premios: o Competición en España:

\begin{verbatim}
-   Coste total: 2,000 € (inscripción) + 1,000 € (transporte) = 3,000 €
-   Premio si gana: 5,000 €
-   Beneficio neto si gana: 5,000 - 3,000 = 2,000 €
-   Beneficio neto si no gana: -3,000 € (pérdida de costes)
\end{verbatim}

o Competición en Alemania:

\begin{verbatim}
- Coste total: 3,500 € (inscripción) + 4,000 € (transporte) = 7,500 €
- Premio si gana: 10,000 €
- Beneficio neto si gana: 10,000 - 7,500 = 2,500 €
- Beneficio neto si no gana: -7,500 € (pérdida de costes)
\end{verbatim}

o Competición en Países Bajos:

\begin{verbatim}
- Coste total: 2,800 € (inscripción) + 3,500 € (transporte) = 6,300 €
- Premio si gana: 7,000 €
- Beneficio neto si gana: 7,000 - 6,300 = 700 €
- Beneficio neto si no gana: -6,300 € (pérdida de costes)
\end{verbatim}

\newpage

\begin{Shaded}
\begin{Highlighting}[]
\NormalTok{tb02}\OtherTok{=} \FunctionTok{crea.tablaX}\NormalTok{(}\FunctionTok{c}\NormalTok{(}\DecValTok{2000}\NormalTok{,}\SpecialCharTok{{-}}\DecValTok{3000}\NormalTok{,}
                    \DecValTok{2500}\NormalTok{,}\SpecialCharTok{{-}}\DecValTok{7500}\NormalTok{,}
                    \DecValTok{700}\NormalTok{,}\SpecialCharTok{{-}}\DecValTok{6300}\NormalTok{),}\AttributeTok{numalternativas =} \DecValTok{3}\NormalTok{,}\AttributeTok{numestados =} \DecValTok{2}\NormalTok{)}
\FunctionTok{rownames}\NormalTok{(tb02)}\OtherTok{=}\FunctionTok{c}\NormalTok{(}\StringTok{"España"}\NormalTok{,}\StringTok{"Alemania"}\NormalTok{,}\StringTok{"Paises Bajos"}\NormalTok{)}
\FunctionTok{colnames}\NormalTok{(tb02)}\OtherTok{=}\FunctionTok{c}\NormalTok{(}\StringTok{"Ganar"}\NormalTok{,}\StringTok{"Perder"}\NormalTok{)}
\NormalTok{knitr}\SpecialCharTok{::}\FunctionTok{kable}\NormalTok{(tb02)}
\end{Highlighting}
\end{Shaded}

\begin{longtable}[]{@{}lrr@{}}
\toprule\noalign{}
& Ganar & Perder \\
\midrule\noalign{}
\endhead
\bottomrule\noalign{}
\endlastfoot
España & 2000 & -3000 \\
Alemania & 2500 & -7500 \\
Paises Bajos & 700 & -6300 \\
\end{longtable}

\begin{Shaded}
\begin{Highlighting}[]
\NormalTok{solp2}\OtherTok{=}\FunctionTok{criterio.Todos}\NormalTok{(tb02,}\AttributeTok{alfa =} \FloatTok{0.3}\NormalTok{,}\AttributeTok{favorable =} \ConstantTok{FALSE}\NormalTok{)}
\NormalTok{knitr}\SpecialCharTok{::}\FunctionTok{kable}\NormalTok{(solp2[,}\DecValTok{3}\SpecialCharTok{:}\FunctionTok{ncol}\NormalTok{(solp2)])}
\end{Highlighting}
\end{Shaded}

\begin{longtable}[]{@{}
  >{\raggedright\arraybackslash}p{(\columnwidth - 12\tabcolsep) * \real{0.2021}}
  >{\raggedright\arraybackslash}p{(\columnwidth - 12\tabcolsep) * \real{0.1383}}
  >{\raggedright\arraybackslash}p{(\columnwidth - 12\tabcolsep) * \real{0.1064}}
  >{\raggedright\arraybackslash}p{(\columnwidth - 12\tabcolsep) * \real{0.1383}}
  >{\raggedright\arraybackslash}p{(\columnwidth - 12\tabcolsep) * \real{0.1383}}
  >{\raggedright\arraybackslash}p{(\columnwidth - 12\tabcolsep) * \real{0.1383}}
  >{\raggedright\arraybackslash}p{(\columnwidth - 12\tabcolsep) * \real{0.1383}}@{}}
\toprule\noalign{}
\begin{minipage}[b]{\linewidth}\raggedright
\end{minipage} & \begin{minipage}[b]{\linewidth}\raggedright
Wald
\end{minipage} & \begin{minipage}[b]{\linewidth}\raggedright
Optimista
\end{minipage} & \begin{minipage}[b]{\linewidth}\raggedright
Hurwicz
\end{minipage} & \begin{minipage}[b]{\linewidth}\raggedright
Savage
\end{minipage} & \begin{minipage}[b]{\linewidth}\raggedright
Laplace
\end{minipage} & \begin{minipage}[b]{\linewidth}\raggedright
Punto Ideal
\end{minipage} \\
\midrule\noalign{}
\endhead
\bottomrule\noalign{}
\endlastfoot
España & 2000 & -3000 & 500 & 4500 & -500 & 4684 \\
Alemania & 2500 & -7500 & -500 & 1800 & -2500 & 1800 \\
Paises Bajos & 700 & -6300 & -1400 & 1200 & -2800 & 1200 \\
iAlt.Opt (Desfav.) & Paises Bajos & Alemania & Paises Bajos & Paises
Bajos & Paises Bajos & Paises Bajos \\
\end{longtable}

\newpage

\section{Roberto González Lozano}\label{roberto-gonzuxe1lez-lozano}

\subsection{Problema 1:}\label{problema-1-1}

Aplicar los criterios de decisión bajo incertidumbre a los problemas
cuya matriz de valores numéricos vienen dadas en la tabla siguiente bajo
dos puntos de vistas:

\begin{Shaded}
\begin{Highlighting}[]
\NormalTok{tablaSCG}\OtherTok{=}\FunctionTok{crea.tablaX}\NormalTok{(}\FunctionTok{c}\NormalTok{(}\DecValTok{2}\NormalTok{,}\DecValTok{5}\NormalTok{,}\DecValTok{7}\NormalTok{,}\DecValTok{2}\NormalTok{,}\SpecialCharTok{{-}}\DecValTok{2}\NormalTok{,}\DecValTok{9}\NormalTok{,}\DecValTok{5}\NormalTok{,}\DecValTok{3}\NormalTok{,}\DecValTok{4}\NormalTok{,}\DecValTok{5}\NormalTok{,}\DecValTok{3}\NormalTok{,}\DecValTok{1}\NormalTok{), }\AttributeTok{numalternativas =} \DecValTok{3}\NormalTok{,}\AttributeTok{numestados =} \DecValTok{4}\NormalTok{)}
\NormalTok{tablaSCG}
\end{Highlighting}
\end{Shaded}

\begin{verbatim}
##    e1 e2 e3 e4
## d1  2  5  7  2
## d2 -2  9  5  3
## d3  4  5  3  1
\end{verbatim}

\begin{enumerate}
\def\labelenumi{\arabic{enumi})}
\tightlist
\item
  \textbf{Caso Favorable (beneficios)}
\item
  \textbf{Caso Desfavorable (costes)}
\end{enumerate}

\subsubsection{Caso Favorable
(beneficios)}\label{caso-favorable-beneficios}

\begin{Shaded}
\begin{Highlighting}[]
\CommentTok{\#Criterio de Wald}

\NormalTok{waldSCG }\OtherTok{=} \FunctionTok{criterio.Wald}\NormalTok{(tablaSCG,T)}
\NormalTok{waldSCG}
\end{Highlighting}
\end{Shaded}

\begin{verbatim}
## $criterio
## [1] "Wald"
## 
## $metodo
## [1] "favorable"
## 
## $tablaX
##    e1 e2 e3 e4
## d1  2  5  7  2
## d2 -2  9  5  3
## d3  4  5  3  1
## 
## $ValorAlternativas
## d1 d2 d3 
##  2 -2  1 
## 
## $ValorOptimo
## [1] 2
## 
## $AlternativaOptima
## d1 
##  1
\end{verbatim}

\begin{Shaded}
\begin{Highlighting}[]
\CommentTok{\# Criterio Optimista}

\NormalTok{optimistaSCG }\OtherTok{=} \FunctionTok{criterio.Optimista}\NormalTok{(tablaSCG,T)}
\NormalTok{optimistaSCG}
\end{Highlighting}
\end{Shaded}

\begin{verbatim}
## $criterio
## [1] "Optimista"
## 
## $metodo
## [1] "favorable"
## 
## $tablaX
##    e1 e2 e3 e4
## d1  2  5  7  2
## d2 -2  9  5  3
## d3  4  5  3  1
## 
## $ValorAlternativas
## d1 d2 d3 
##  7  9  5 
## 
## $ValorOptimo
## [1] 9
## 
## $AlternativaOptima
## d2 
##  2
\end{verbatim}

\begin{Shaded}
\begin{Highlighting}[]
\CommentTok{\# Criterio Hurwicz}

\NormalTok{hurwiczSCG }\OtherTok{=} \FunctionTok{criterio.Hurwicz}\NormalTok{(tablaSCG, }\FloatTok{0.5}\NormalTok{, T)}
\NormalTok{hurwiczSCG}
\end{Highlighting}
\end{Shaded}

\begin{verbatim}
## $criterio
## [1] "Hurwicz"
## 
## $alfa
## [1] 0.5
## 
## $metodo
## [1] "favorable"
## 
## $tablaX
##    e1 e2 e3 e4
## d1  2  5  7  2
## d2 -2  9  5  3
## d3  4  5  3  1
## 
## $ValorAlternativas
##  d1  d2  d3 
## 4.5 3.5 3.0 
## 
## $ValorOptimo
## [1] 4.5
## 
## $AlternativaOptima
## d1 
##  1
\end{verbatim}

\begin{Shaded}
\begin{Highlighting}[]
\CommentTok{\# Criterio Savage}

\NormalTok{savageSCG }\OtherTok{=} \FunctionTok{criterio.Savage}\NormalTok{(tablaSCG,T)}
\NormalTok{savageSCG}
\end{Highlighting}
\end{Shaded}

\begin{verbatim}
## $criterio
## [1] "Savage"
## 
## $metodo
## [1] "favorable"
## 
## $tablaX
##    e1 e2 e3 e4
## d1  2  5  7  2
## d2 -2  9  5  3
## d3  4  5  3  1
## 
## $Mejores
## e1 e2 e3 e4 
##  4  9  7  3 
## 
## $Pesos
##    e1 e2 e3 e4
## d1  2  4  0  1
## d2  6  0  2  0
## d3  0  4  4  2
## 
## $ValorAlternativas
## d1 d2 d3 
##  4  6  4 
## 
## $ValorOptimo
## [1] 4
## 
## $AlternativaOptima
## d1 d3 
##  1  3
\end{verbatim}

\begin{Shaded}
\begin{Highlighting}[]
\CommentTok{\# Criterio Laplace}

\NormalTok{laplaceSCG }\OtherTok{=} \FunctionTok{criterio.Laplace}\NormalTok{(tablaSCG,T)}
\NormalTok{laplaceSCG}
\end{Highlighting}
\end{Shaded}

\begin{verbatim}
## $criterio
## [1] "Laplace"
## 
## $metodo
## [1] "favorable"
## 
## $tablaX
##    e1 e2 e3 e4
## d1  2  5  7  2
## d2 -2  9  5  3
## d3  4  5  3  1
## 
## $ValorAlternativas
##   d1   d2   d3 
## 4.00 3.75 3.25 
## 
## $ValorOptimo
## [1] 4
## 
## $AlternativaOptima
## d1 
##  1
\end{verbatim}

\begin{Shaded}
\begin{Highlighting}[]
\CommentTok{\# Criterio Punto Ideal}

\NormalTok{pidealSCG }\OtherTok{=} \FunctionTok{criterio.PuntoIdeal}\NormalTok{(tablaSCG,T)}
\NormalTok{pidealSCG}
\end{Highlighting}
\end{Shaded}

\begin{verbatim}
## $criterio
## [1] "Punto Ideal"
## 
## $metodo
## [1] "favorable"
## 
## $tablaX
##    e1 e2 e3 e4
## d1  2  5  7  2
## d2 -2  9  5  3
## d3  4  5  3  1
## 
## $Mejores
## e1 e2 e3 e4 
##  4  9  7  3 
## 
## $ValorAlternativas
##       d1       d2       d3 
## 4.582576 6.324555 6.000000 
## 
## $ValorOptimo
## [1] 4.582576
## 
## $AlternativaOptima
## d1 
##  1
\end{verbatim}

\begin{Shaded}
\begin{Highlighting}[]
\CommentTok{\#Todos los criterios}

\NormalTok{todosSCG }\OtherTok{=} \FunctionTok{criterio.Todos}\NormalTok{(tablaSCG, }\AttributeTok{alfa=}\FloatTok{0.5}\NormalTok{, }\AttributeTok{favorable =}\NormalTok{ T)}
\NormalTok{todosSCG}
\end{Highlighting}
\end{Shaded}

\begin{verbatim}
##                 e1 e2 e3 e4 Wald Optimista Hurwicz Savage Laplace Punto Ideal
## d1               2  5  7  2    2         7     4.5      4    4.00       4.583
## d2              -2  9  5  3   -2         9     3.5      6    3.75       6.325
## d3               4  5  3  1    1         5     3.0      4    3.25       6.000
## iAlt.Opt (fav.) -- -- -- --   d1        d2      d1  d1,d3      d1          d1
\end{verbatim}

\textbf{Por tanto, en caso favorable, selecciono la alternativa d1.}

\subsubsection{Caso Desfavorable
(costes)}\label{caso-desfavorable-costes}

\begin{Shaded}
\begin{Highlighting}[]
\CommentTok{\#Criterio de Wald}

\NormalTok{waldSCG2 }\OtherTok{=} \FunctionTok{criterio.Wald}\NormalTok{(tablaSCG,F)}
\NormalTok{waldSCG2}
\end{Highlighting}
\end{Shaded}

\begin{verbatim}
## $criterio
## [1] "Wald"
## 
## $metodo
## [1] "desfavorable"
## 
## $tablaX
##    e1 e2 e3 e4
## d1  2  5  7  2
## d2 -2  9  5  3
## d3  4  5  3  1
## 
## $ValorAlternativas
## d1 d2 d3 
##  7  9  5 
## 
## $ValorOptimo
## [1] 5
## 
## $AlternativaOptima
## d3 
##  3
\end{verbatim}

\begin{Shaded}
\begin{Highlighting}[]
\CommentTok{\# Criterio Optimista}

\NormalTok{optimistaSCG2 }\OtherTok{=} \FunctionTok{criterio.Optimista}\NormalTok{(tablaSCG,F)}
\NormalTok{optimistaSCG2}
\end{Highlighting}
\end{Shaded}

\begin{verbatim}
## $criterio
## [1] "Optimista"
## 
## $metodo
## [1] "desfavorable"
## 
## $tablaX
##    e1 e2 e3 e4
## d1  2  5  7  2
## d2 -2  9  5  3
## d3  4  5  3  1
## 
## $ValorAlternativas
## d1 d2 d3 
##  2 -2  1 
## 
## $ValorOptimo
## [1] -2
## 
## $AlternativaOptima
## d2 
##  2
\end{verbatim}

\begin{Shaded}
\begin{Highlighting}[]
\CommentTok{\# Criterio Hurwicz}

\NormalTok{hurwiczSCG2 }\OtherTok{=} \FunctionTok{criterio.Hurwicz}\NormalTok{(tablaSCG, }\FloatTok{0.5}\NormalTok{, F)}
\NormalTok{hurwiczSCG2}
\end{Highlighting}
\end{Shaded}

\begin{verbatim}
## $criterio
## [1] "Hurwicz"
## 
## $alfa
## [1] 0.5
## 
## $metodo
## [1] "desfavorable"
## 
## $tablaX
##    e1 e2 e3 e4
## d1  2  5  7  2
## d2 -2  9  5  3
## d3  4  5  3  1
## 
## $ValorAlternativas
##  d1  d2  d3 
## 4.5 3.5 3.0 
## 
## $ValorOptimo
## [1] 3
## 
## $AlternativaOptima
## d3 
##  3
\end{verbatim}

\begin{Shaded}
\begin{Highlighting}[]
\CommentTok{\# Criterio Savage}

\NormalTok{savageSCG2 }\OtherTok{=} \FunctionTok{criterio.Savage}\NormalTok{(tablaSCG,F)}
\NormalTok{savageSCG2}
\end{Highlighting}
\end{Shaded}

\begin{verbatim}
## $criterio
## [1] "Savage"
## 
## $metodo
## [1] "desfavorable"
## 
## $tablaX
##    e1 e2 e3 e4
## d1  2  5  7  2
## d2 -2  9  5  3
## d3  4  5  3  1
## 
## $Mejores
## e1 e2 e3 e4 
## -2  5  3  1 
## 
## $Pesos
##    e1 e2 e3 e4
## d1  4  0  4  1
## d2  0  4  2  2
## d3  6  0  0  0
## 
## $ValorAlternativas
## d1 d2 d3 
##  4  4  6 
## 
## $ValorOptimo
## [1] 4
## 
## $AlternativaOptima
## d1 d2 
##  1  2
\end{verbatim}

\begin{Shaded}
\begin{Highlighting}[]
\CommentTok{\# Criterio Laplace}

\NormalTok{laplaceSCG2 }\OtherTok{=} \FunctionTok{criterio.Laplace}\NormalTok{(tablaSCG,F)}
\NormalTok{laplaceSCG2}
\end{Highlighting}
\end{Shaded}

\begin{verbatim}
## $criterio
## [1] "Laplace"
## 
## $metodo
## [1] "desfavorable"
## 
## $tablaX
##    e1 e2 e3 e4
## d1  2  5  7  2
## d2 -2  9  5  3
## d3  4  5  3  1
## 
## $ValorAlternativas
##   d1   d2   d3 
## 4.00 3.75 3.25 
## 
## $ValorOptimo
## [1] 3.25
## 
## $AlternativaOptima
## d3 
##  3
\end{verbatim}

\begin{Shaded}
\begin{Highlighting}[]
\CommentTok{\# Criterio Punto Ideal}

\NormalTok{pidealSCG2 }\OtherTok{=} \FunctionTok{criterio.PuntoIdeal}\NormalTok{(tablaSCG,F)}
\NormalTok{pidealSCG2}
\end{Highlighting}
\end{Shaded}

\begin{verbatim}
## $criterio
## [1] "Punto Ideal"
## 
## $metodo
## [1] "desfavorable"
## 
## $tablaX
##    e1 e2 e3 e4
## d1  2  5  7  2
## d2 -2  9  5  3
## d3  4  5  3  1
## 
## $Mejores
## e1 e2 e3 e4 
## -2  5  3  1 
## 
## $ValorAlternativas
##       d1       d2       d3 
## 5.744563 4.898979 6.000000 
## 
## $ValorOptimo
## [1] 4.898979
## 
## $AlternativaOptima
## d2 
##  2
\end{verbatim}

\begin{Shaded}
\begin{Highlighting}[]
\CommentTok{\#Todos los criterios}

\NormalTok{todosSCG2 }\OtherTok{=} \FunctionTok{criterio.Todos}\NormalTok{(tablaSCG, }\AttributeTok{alfa=}\FloatTok{0.5}\NormalTok{, }\AttributeTok{favorable =}\NormalTok{ F)}
\NormalTok{todosSCG2}
\end{Highlighting}
\end{Shaded}

\begin{verbatim}
##                    e1 e2 e3 e4 Wald Optimista Hurwicz Savage Laplace
## d1                  2  5  7  2    7         2     4.5      4    4.00
## d2                 -2  9  5  3    9        -2     3.5      4    3.75
## d3                  4  5  3  1    5         1     3.0      6    3.25
## iAlt.Opt (Desfav.) -- -- -- --   d3        d2      d3  d1,d2      d3
##                    Punto Ideal
## d1                       5.745
## d2                       4.899
## d3                       6.000
## iAlt.Opt (Desfav.)          d2
\end{verbatim}

\textbf{Por tanto, en caso desfavorable, selecciono la alternativa d2.}

\subsection{Problema 2:}\label{problema-2-1}

Luis De La Fuente, seleccionador internacional de España, debe presentar
la lista de futbolistas convocados para la posiciones de delanteros de
cara al Mundial 2026. Debe llevar 5 personas, y hasta ahora ya tiene 4
seleccionadas. Para el último puesto que queda disponible, tiene tres
futbolistas candidatas, los cuales tienen puntuaciones en cada uno de
estas habilidades (pase, velocidad y gol, calculadas sobre 100):

\begin{itemize}
\item
  \textbf{Ayoze} es un jugador del Villarreal que en pase tiene una
  puntuación de 55, en velocidad 78 y en gol 81.
\item
  \textbf{Isi Palazón}, del Rayo Vallecano, tiene unas puntuaciones de
  66 en pase, 77 en velocidad y 74 en gol.
\item
  \textbf{Samu Orodion}, del Oporto, tiene puntuación de 80 en pase, 62
  en velocidad y 73 en gol.
\end{itemize}

Decidir qué futbolista debe ser convocado para el Mundial 2028.

\textbf{Solución}

1. Crear la matriz de datos:

\begin{Shaded}
\begin{Highlighting}[]
\NormalTok{datosSCG }\OtherTok{\textless{}{-}} \FunctionTok{matrix}\NormalTok{(}\AttributeTok{data =} \FunctionTok{c}\NormalTok{(}\DecValTok{55}\NormalTok{,}\DecValTok{78}\NormalTok{,}\DecValTok{81}\NormalTok{,}\DecValTok{66}\NormalTok{,}\DecValTok{77}\NormalTok{,}\DecValTok{74}\NormalTok{,}\DecValTok{80}\NormalTok{,}\DecValTok{62}\NormalTok{,}\DecValTok{73}\NormalTok{), }\AttributeTok{nrow =} \DecValTok{3}\NormalTok{, }\AttributeTok{byrow =}\NormalTok{ T)}
\FunctionTok{rownames}\NormalTok{(datosSCG) }\OtherTok{\textless{}{-}} \FunctionTok{c}\NormalTok{(}\StringTok{"Ayoze"}\NormalTok{, }\StringTok{"Isi Palazón"}\NormalTok{, }\StringTok{"Samu Orodion"}\NormalTok{)}
\FunctionTok{colnames}\NormalTok{(datosSCG) }\OtherTok{\textless{}{-}} \FunctionTok{c}\NormalTok{(}\StringTok{"pase"}\NormalTok{, }\StringTok{"velocidad"}\NormalTok{, }\StringTok{"gol"}\NormalTok{)}
\NormalTok{datosSCG}
\end{Highlighting}
\end{Shaded}

\begin{verbatim}
##              pase velocidad gol
## Ayoze          55        78  81
## Isi Palazón    66        77  74
## Samu Orodion   80        62  73
\end{verbatim}

Estamos ante un problema de decisión de un solo decisor, que tiene un
modelo de beneficios (caso favorable). Con los datos de la tabla
anterior:

\begin{Shaded}
\begin{Highlighting}[]
\CommentTok{\#Criterio de Wald}

\NormalTok{waldSCG }\OtherTok{=} \FunctionTok{criterio.Wald}\NormalTok{(datosSCG,T)}
\NormalTok{waldSCG}
\end{Highlighting}
\end{Shaded}

\begin{verbatim}
## $criterio
## [1] "Wald"
## 
## $metodo
## [1] "favorable"
## 
## $tablaX
##              pase velocidad gol
## Ayoze          55        78  81
## Isi Palazón    66        77  74
## Samu Orodion   80        62  73
## 
## $ValorAlternativas
##        Ayoze  Isi Palazón Samu Orodion 
##           55           66           62 
## 
## $ValorOptimo
## [1] 66
## 
## $AlternativaOptima
## Isi Palazón 
##           2
\end{verbatim}

\begin{Shaded}
\begin{Highlighting}[]
\CommentTok{\# Criterio Optimista}

\NormalTok{optimistaSCG }\OtherTok{=} \FunctionTok{criterio.Optimista}\NormalTok{(datosSCG,T)}
\NormalTok{optimistaSCG}
\end{Highlighting}
\end{Shaded}

\begin{verbatim}
## $criterio
## [1] "Optimista"
## 
## $metodo
## [1] "favorable"
## 
## $tablaX
##              pase velocidad gol
## Ayoze          55        78  81
## Isi Palazón    66        77  74
## Samu Orodion   80        62  73
## 
## $ValorAlternativas
##        Ayoze  Isi Palazón Samu Orodion 
##           81           77           80 
## 
## $ValorOptimo
## [1] 81
## 
## $AlternativaOptima
## Ayoze 
##     1
\end{verbatim}

\begin{Shaded}
\begin{Highlighting}[]
\CommentTok{\# Criterio Hurwicz}

\NormalTok{hurwiczSCG }\OtherTok{=} \FunctionTok{criterio.Hurwicz}\NormalTok{(datosSCG, }\FloatTok{0.5}\NormalTok{, T)}
\NormalTok{hurwiczSCG}
\end{Highlighting}
\end{Shaded}

\begin{verbatim}
## $criterio
## [1] "Hurwicz"
## 
## $alfa
## [1] 0.5
## 
## $metodo
## [1] "favorable"
## 
## $tablaX
##              pase velocidad gol
## Ayoze          55        78  81
## Isi Palazón    66        77  74
## Samu Orodion   80        62  73
## 
## $ValorAlternativas
##        Ayoze  Isi Palazón Samu Orodion 
##         68.0         71.5         71.0 
## 
## $ValorOptimo
## [1] 71.5
## 
## $AlternativaOptima
## Isi Palazón 
##           2
\end{verbatim}

\begin{Shaded}
\begin{Highlighting}[]
\CommentTok{\# Criterio Savage}

\NormalTok{savageSCG }\OtherTok{=} \FunctionTok{criterio.Savage}\NormalTok{(datosSCG,T)}
\NormalTok{savageSCG}
\end{Highlighting}
\end{Shaded}

\begin{verbatim}
## $criterio
## [1] "Savage"
## 
## $metodo
## [1] "favorable"
## 
## $tablaX
##              pase velocidad gol
## Ayoze          55        78  81
## Isi Palazón    66        77  74
## Samu Orodion   80        62  73
## 
## $Mejores
##      pase velocidad       gol 
##        80        78        81 
## 
## $Pesos
##              pase velocidad gol
## Ayoze          25         0   0
## Isi Palazón    14         1   7
## Samu Orodion    0        16   8
## 
## $ValorAlternativas
##        Ayoze  Isi Palazón Samu Orodion 
##           25           14           16 
## 
## $ValorOptimo
## [1] 14
## 
## $AlternativaOptima
## Isi Palazón 
##           2
\end{verbatim}

\begin{Shaded}
\begin{Highlighting}[]
\CommentTok{\# Criterio Laplace}

\NormalTok{laplaceSCG }\OtherTok{=} \FunctionTok{criterio.Laplace}\NormalTok{(datosSCG,T)}
\NormalTok{laplaceSCG}
\end{Highlighting}
\end{Shaded}

\begin{verbatim}
## $criterio
## [1] "Laplace"
## 
## $metodo
## [1] "favorable"
## 
## $tablaX
##              pase velocidad gol
## Ayoze          55        78  81
## Isi Palazón    66        77  74
## Samu Orodion   80        62  73
## 
## $ValorAlternativas
##        Ayoze  Isi Palazón Samu Orodion 
##     71.33333     72.33333     71.66667 
## 
## $ValorOptimo
## [1] 72.33333
## 
## $AlternativaOptima
## Isi Palazón 
##           2
\end{verbatim}

\begin{Shaded}
\begin{Highlighting}[]
\CommentTok{\# Criterio Punto Ideal}

\NormalTok{pidealSCG }\OtherTok{=} \FunctionTok{criterio.PuntoIdeal}\NormalTok{(datosSCG,T)}
\NormalTok{pidealSCG}
\end{Highlighting}
\end{Shaded}

\begin{verbatim}
## $criterio
## [1] "Punto Ideal"
## 
## $metodo
## [1] "favorable"
## 
## $tablaX
##              pase velocidad gol
## Ayoze          55        78  81
## Isi Palazón    66        77  74
## Samu Orodion   80        62  73
## 
## $Mejores
##      pase velocidad       gol 
##        80        78        81 
## 
## $ValorAlternativas
##        Ayoze  Isi Palazón Samu Orodion 
##     25.00000     15.68439     17.88854 
## 
## $ValorOptimo
## [1] 15.68439
## 
## $AlternativaOptima
## Isi Palazón 
##           2
\end{verbatim}

\begin{Shaded}
\begin{Highlighting}[]
\CommentTok{\#Todos los criterios}

\NormalTok{todosSCG }\OtherTok{=} \FunctionTok{criterio.Todos}\NormalTok{(datosSCG, }\AttributeTok{alfa=}\FloatTok{0.5}\NormalTok{, }\AttributeTok{favorable =}\NormalTok{ T)}
\NormalTok{todosSCG}
\end{Highlighting}
\end{Shaded}

\begin{verbatim}
##                 pase velocidad gol        Wald Optimista     Hurwicz
## Ayoze             55        78  81          55        81        68.0
## Isi Palazón       66        77  74          66        77        71.5
## Samu Orodion      80        62  73          62        80        71.0
## iAlt.Opt (fav.)   --        --  -- Isi Palazón     Ayoze Isi Palazón
##                      Savage     Laplace Punto Ideal
## Ayoze                    25       71.33       25.00
## Isi Palazón              14       72.33       15.68
## Samu Orodion             16       71.67       17.89
## iAlt.Opt (fav.) Isi Palazón Isi Palazón Isi Palazón
\end{verbatim}

\textbf{\emph{CONCLUSIÓN:}}

\textbf{Todos los criterios empleados recomiendan convocar a Isi
Palazón}, a excepción del criterio Optimista, que recomienda a Ayoze ,
ya que el criterio optimista siempre actúa en la mejor situación
posible, en la que Ayoze, en este caso, haría una gran actuación a nivel
de goles, la cuál sería determinante.

\textbf{Finalmente, se selecciona a Isi Palazón.}

\newpage

\section{Belén Puerta González}\label{beluxe9n-puerta-gonzuxe1lez}

\subsection{Problema 1}\label{problema-1-2}

Aplicar los criterios de decisión bajo incertidumbre a la siguiente
matriz de valores:

\begin{longtable}[]{@{}lrrr@{}}
\toprule\noalign{}
& e1 & e2 & e3 \\
\midrule\noalign{}
\endhead
\bottomrule\noalign{}
\endlastfoot
d1 & 5 & 3 & -1 \\
d2 & 12 & 6 & 6 \\
d3 & 10 & 4 & 8 \\
d4 & 13 & 4 & 1 \\
d5 & 5 & 8 & 10 \\
\end{longtable}

Resolver primero para el caso FAVORABLE (beneficios) y, posteriormente,
para el caso DESFAVORABLE (costos).

\subsubsection{CASO FAVORABLE}\label{caso-favorable}

\textbf{1. Criterio de Wald}

\begin{Shaded}
\begin{Highlighting}[]
\NormalTok{p1\_pes\_FAV }\OtherTok{=} \FunctionTok{criterio.Wald}\NormalTok{(datos,T)}
\FunctionTok{names}\NormalTok{(p1\_pes\_FAV}\SpecialCharTok{$}\NormalTok{AlternativaOptima) }
\end{Highlighting}
\end{Shaded}

\begin{verbatim}
## [1] "d2"
\end{verbatim}

\begin{Shaded}
\begin{Highlighting}[]
\CommentTok{\# La mejor decisión es la alternativa d2.}
\end{Highlighting}
\end{Shaded}

\textbf{2. Criterio optimista}

\begin{Shaded}
\begin{Highlighting}[]
\NormalTok{p1\_opt\_FAV }\OtherTok{=} \FunctionTok{criterio.Optimista}\NormalTok{(datos,T)}
\FunctionTok{names}\NormalTok{(p1\_opt\_FAV}\SpecialCharTok{$}\NormalTok{AlternativaOptima) }
\end{Highlighting}
\end{Shaded}

\begin{verbatim}
## [1] "d4"
\end{verbatim}

\begin{Shaded}
\begin{Highlighting}[]
\CommentTok{\# La mejor decisión es la alternativa d4.}
\end{Highlighting}
\end{Shaded}

\textbf{3. Criterio de Hurwitcz}

\begin{Shaded}
\begin{Highlighting}[]
\NormalTok{p1\_hur\_FAV }\OtherTok{=} \FunctionTok{criterio.Hurwicz}\NormalTok{(datos,T)}
\FunctionTok{names}\NormalTok{(p1\_hur\_FAV}\SpecialCharTok{$}\NormalTok{AlternativaOptima) }
\end{Highlighting}
\end{Shaded}

\begin{verbatim}
## [1] "d4"
\end{verbatim}

\begin{Shaded}
\begin{Highlighting}[]
\CommentTok{\# La mejor decisión es la alternativa d4.}
\end{Highlighting}
\end{Shaded}

\begin{Shaded}
\begin{Highlighting}[]
\FunctionTok{dibuja.criterio.Hurwicz}\NormalTok{(datos, T)}
\end{Highlighting}
\end{Shaded}

\includegraphics{Resolucion_files/figure-latex/unnamed-chunk-29-1.pdf}

\textbf{4. Criterio de Savage}

\begin{Shaded}
\begin{Highlighting}[]
\NormalTok{p1\_sav\_FAV }\OtherTok{=} \FunctionTok{criterio.Savage}\NormalTok{(datos,T)}
\FunctionTok{names}\NormalTok{(p1\_sav\_FAV}\SpecialCharTok{$}\NormalTok{AlternativaOptima) }
\end{Highlighting}
\end{Shaded}

\begin{verbatim}
## [1] "d2" "d3"
\end{verbatim}

\begin{Shaded}
\begin{Highlighting}[]
\CommentTok{\# La mejor decisión son las alternativas d2 y d3.}
\end{Highlighting}
\end{Shaded}

\textbf{5. Criterio de Laplace}

\begin{Shaded}
\begin{Highlighting}[]
\NormalTok{p1\_lap\_FAV }\OtherTok{=} \FunctionTok{criterio.Laplace}\NormalTok{(datos,T)}
\FunctionTok{names}\NormalTok{(p1\_lap\_FAV}\SpecialCharTok{$}\NormalTok{AlternativaOptima) }
\end{Highlighting}
\end{Shaded}

\begin{verbatim}
## [1] "d2"
\end{verbatim}

\begin{Shaded}
\begin{Highlighting}[]
\CommentTok{\# La mejor decisión es la alternativa d2.}
\end{Highlighting}
\end{Shaded}

\textbf{6. Criterio del punto ideal}

\begin{Shaded}
\begin{Highlighting}[]
\NormalTok{p1\_pid\_FAV }\OtherTok{=} \FunctionTok{criterio.PuntoIdeal}\NormalTok{(datos,T)}
\FunctionTok{names}\NormalTok{(p1\_pid\_FAV}\SpecialCharTok{$}\NormalTok{AlternativaOptima) }
\end{Highlighting}
\end{Shaded}

\begin{verbatim}
## [1] "d2"
\end{verbatim}

\begin{Shaded}
\begin{Highlighting}[]
\CommentTok{\# La mejor decisión es la alternativa d2.}
\end{Highlighting}
\end{Shaded}

\textbf{\emph{CONCLUSIÓN FINAL:}} En el escenario favorable, es decir,
considerando que los datos son beneficios, podemos observar que
mayoritariamente la alternativa d2 es la más elegida.

\subsubsection{CASO DESFAVORABLE}\label{caso-desfavorable}

\textbf{1. Criterio de Wald}

\begin{Shaded}
\begin{Highlighting}[]
\NormalTok{p1\_pes\_DESFAV }\OtherTok{=} \FunctionTok{criterio.Wald}\NormalTok{(datos,F)}
\FunctionTok{names}\NormalTok{(p1\_pes\_DESFAV}\SpecialCharTok{$}\NormalTok{AlternativaOptima) }
\end{Highlighting}
\end{Shaded}

\begin{verbatim}
## [1] "d1"
\end{verbatim}

\begin{Shaded}
\begin{Highlighting}[]
\CommentTok{\# La mejor decisión es la alternativa d1.}
\end{Highlighting}
\end{Shaded}

\textbf{2. Criterio optimista}

\begin{Shaded}
\begin{Highlighting}[]
\NormalTok{p1\_opt\_DESFAV }\OtherTok{=} \FunctionTok{criterio.Optimista}\NormalTok{(datos,F)}
\FunctionTok{names}\NormalTok{(p1\_opt\_DESFAV}\SpecialCharTok{$}\NormalTok{AlternativaOptima) }
\end{Highlighting}
\end{Shaded}

\begin{verbatim}
## [1] "d1"
\end{verbatim}

\begin{Shaded}
\begin{Highlighting}[]
\CommentTok{\# La mejor decisión es la alternativa d1.}
\end{Highlighting}
\end{Shaded}

\textbf{3. Criterio de Hurwitcz}

\begin{Shaded}
\begin{Highlighting}[]
\NormalTok{p1\_hur\_DESFAV }\OtherTok{=} \FunctionTok{criterio.Hurwicz}\NormalTok{(datos,F)}
\FunctionTok{names}\NormalTok{(p1\_hur\_DESFAV}\SpecialCharTok{$}\NormalTok{AlternativaOptima) }
\end{Highlighting}
\end{Shaded}

\begin{verbatim}
## [1] "d2"
\end{verbatim}

\begin{Shaded}
\begin{Highlighting}[]
\CommentTok{\# La mejor decisión es la alternativa d2.}
\end{Highlighting}
\end{Shaded}

\begin{Shaded}
\begin{Highlighting}[]
\FunctionTok{dibuja.criterio.Hurwicz}\NormalTok{(datos,F)}
\end{Highlighting}
\end{Shaded}

\includegraphics{Resolucion_files/figure-latex/unnamed-chunk-36-1.pdf}

\textbf{4. Criterio de Savage}

\begin{Shaded}
\begin{Highlighting}[]
\NormalTok{p1\_sav\_DESFAV }\OtherTok{=} \FunctionTok{criterio.Savage}\NormalTok{(datos,F)}
\FunctionTok{names}\NormalTok{(p1\_sav\_DESFAV}\SpecialCharTok{$}\NormalTok{AlternativaOptima) }
\end{Highlighting}
\end{Shaded}

\begin{verbatim}
## [1] "d1"
\end{verbatim}

\begin{Shaded}
\begin{Highlighting}[]
\CommentTok{\# La mejor decisión es la alternativa d1.}
\end{Highlighting}
\end{Shaded}

\textbf{5. Criterio de Laplace}

\begin{Shaded}
\begin{Highlighting}[]
\NormalTok{p1\_lap\_DESFAV }\OtherTok{=} \FunctionTok{criterio.Laplace}\NormalTok{(datos,F)}
\FunctionTok{names}\NormalTok{(p1\_lap\_DESFAV}\SpecialCharTok{$}\NormalTok{AlternativaOptima) }
\end{Highlighting}
\end{Shaded}

\begin{verbatim}
## [1] "d1"
\end{verbatim}

\begin{Shaded}
\begin{Highlighting}[]
\CommentTok{\# La mejor decisión es la alternativa d1.}
\end{Highlighting}
\end{Shaded}

\textbf{6. Criterio del punto ideal}

\begin{Shaded}
\begin{Highlighting}[]
\NormalTok{p1\_pid\_DESFAV }\OtherTok{=} \FunctionTok{criterio.PuntoIdeal}\NormalTok{(datos,F)}
\FunctionTok{names}\NormalTok{(p1\_pid\_DESFAV}\SpecialCharTok{$}\NormalTok{AlternativaOptima) }
\end{Highlighting}
\end{Shaded}

\begin{verbatim}
## [1] "d1"
\end{verbatim}

\begin{Shaded}
\begin{Highlighting}[]
\CommentTok{\# La mejor decisión es la alternativa d1.}
\end{Highlighting}
\end{Shaded}

\textbf{\emph{CONCLUSIÓN FINAL:}} En el escenario desfavorable, es
decir, considerando que los datos son costes, podemos observar que
mayoritariamente la alternativa d1 es la más elegida.

\subsection{Problema 2}\label{problema-2-2}

Un inversionista ha decidido analizar diversas opciones de inversión, ya
que busca maximizar sus ganancias bajo distintos escenarios del mercado.
Tras meses de estudio y consultas con expertos financieros, ha reducido
sus opciones a cuatro activos financieros que, según su análisis,
presentan distintas oportunidades de rendimiento.

Ha identificado tres posibles escenarios del mercado: mercado alcista,
en el que se espera un crecimiento considerable de la economía; mercado
estable, donde los rendimientos se mantendrán sin grandes cambios; y
mercado bajista, en el que la economía enfrentará dificultades.

Con base en las proyecciones actuales, los rendimientos estimados para
cada activo bajo estos escenarios, expresados en millones de euros, son
los siguientes:

\begin{itemize}
\item
  \textbf{Acciones tecnológicas}: en un escenario alcista generarían 15
  millones, en un mercado estable producirían 7 millones, y en un
  mercado bajista solo darían 1 millón. Este tipo de activo se ha
  caracterizadompor su enorme potencial en épocas de bonanza.
\item
  \textbf{Bonos del gobierno}: se proyecta que generen un rendimiento
  constante de 5 millones sin importar el estado del mercado. Se
  consideran una opción de bajo riesgo, especialmente en tiempos de
  incertidumbre, debido a su estabilidad.
\item
  \textbf{Fondos de bienes raíces}: bajo un mercado alcista podrían
  proporcionar un rendimiento de 10 millones, en un mercado estable 6
  millones, y en un escenario bajista 2 millones.
\item
  \textbf{Criptomonedas}: en un mercado alcista su rentabilidad podría
  alcanzar los 25 millones, en un mercado estable solo generarían 3
  millones, y en un mercado bajista apenas 1 millón. Este activo es
  conocido por su alta volatilidad.
\end{itemize}

Dado este panorama y los rendimientos proyectados, ¿cuál sería la opción
más adecuada que recomendarías al inversionista?

\begin{Shaded}
\begin{Highlighting}[]
\NormalTok{datos2 }\OtherTok{=} \FunctionTok{crea.tablaX}\NormalTok{(}\FunctionTok{c}\NormalTok{(}\DecValTok{15}\NormalTok{, }\DecValTok{7}\NormalTok{, }\DecValTok{1}\NormalTok{,}
                       \DecValTok{5}\NormalTok{,  }\DecValTok{5}\NormalTok{, }\DecValTok{5}\NormalTok{,}
                       \DecValTok{10}\NormalTok{, }\DecValTok{6}\NormalTok{, }\DecValTok{2}\NormalTok{,}
                       \DecValTok{25}\NormalTok{, }\DecValTok{3}\NormalTok{, }\DecValTok{1}\NormalTok{), }\AttributeTok{numalternativas =} \DecValTok{4}\NormalTok{, }\AttributeTok{numestados =} \DecValTok{3}\NormalTok{)}
\NormalTok{knitr}\SpecialCharTok{::}\FunctionTok{kable}\NormalTok{(datos2)}
\end{Highlighting}
\end{Shaded}

\begin{longtable}[]{@{}lrrr@{}}
\toprule\noalign{}
& e1 & e2 & e3 \\
\midrule\noalign{}
\endhead
\bottomrule\noalign{}
\endlastfoot
d1 & 15 & 7 & 1 \\
d2 & 5 & 5 & 5 \\
d3 & 10 & 6 & 2 \\
d4 & 25 & 3 & 1 \\
\end{longtable}

\begin{Shaded}
\begin{Highlighting}[]
\NormalTok{Alternativa }\OtherTok{=} \FunctionTok{criterio.Todos}\NormalTok{(datos2, }\AttributeTok{favorable =}\NormalTok{ T) }
\CommentTok{\# El inversionista tiene margen de elección para buscar maximizar }
\CommentTok{\# ganancias, lo que indica un problema en una situación favorable.}
\end{Highlighting}
\end{Shaded}

\begin{Shaded}
\begin{Highlighting}[]
\NormalTok{knitr}\SpecialCharTok{::}\FunctionTok{kable}\NormalTok{(Alternativa[,}\DecValTok{4}\SpecialCharTok{:}\FunctionTok{ncol}\NormalTok{(Alternativa)])}
\end{Highlighting}
\end{Shaded}

\begin{longtable}[]{@{}
  >{\raggedright\arraybackslash}p{(\columnwidth - 12\tabcolsep) * \real{0.2424}}
  >{\raggedright\arraybackslash}p{(\columnwidth - 12\tabcolsep) * \real{0.0758}}
  >{\raggedright\arraybackslash}p{(\columnwidth - 12\tabcolsep) * \real{0.1515}}
  >{\raggedright\arraybackslash}p{(\columnwidth - 12\tabcolsep) * \real{0.1212}}
  >{\raggedright\arraybackslash}p{(\columnwidth - 12\tabcolsep) * \real{0.1061}}
  >{\raggedright\arraybackslash}p{(\columnwidth - 12\tabcolsep) * \real{0.1212}}
  >{\raggedright\arraybackslash}p{(\columnwidth - 12\tabcolsep) * \real{0.1818}}@{}}
\toprule\noalign{}
\begin{minipage}[b]{\linewidth}\raggedright
\end{minipage} & \begin{minipage}[b]{\linewidth}\raggedright
Wald
\end{minipage} & \begin{minipage}[b]{\linewidth}\raggedright
Optimista
\end{minipage} & \begin{minipage}[b]{\linewidth}\raggedright
Hurwicz
\end{minipage} & \begin{minipage}[b]{\linewidth}\raggedright
Savage
\end{minipage} & \begin{minipage}[b]{\linewidth}\raggedright
Laplace
\end{minipage} & \begin{minipage}[b]{\linewidth}\raggedright
Punto Ideal
\end{minipage} \\
\midrule\noalign{}
\endhead
\bottomrule\noalign{}
\endlastfoot
d1 & 1 & 15 & 5.2 & 10 & 7.667 & 10.770 \\
d2 & 5 & 5 & 5.0 & 20 & 5.000 & 20.100 \\
d3 & 2 & 10 & 4.4 & 15 & 6.000 & 15.330 \\
d4 & 1 & 25 & 8.2 & 4 & 9.667 & 5.657 \\
iAlt.Opt (fav.) & d2 & d4 & d4 & d4 & d4 & d4 \\
\end{longtable}

\textbf{\emph{CONCLUSIÓN}}: Podemos observar que, mayoritariamente la
mejor alternativa serían las criptomonedas. Tan solo en el criterio de
Wald, se obtiene que la mejor alternativa serían los bonos del gobierno.

\newpage

\section{María del Rosario Ruiz
Avila}\label{maruxeda-del-rosario-ruiz-avila}

\subsection{Problema 1}\label{problema-1-3}

Aplicar los criterios de decisión bajo incertidumbre a los problemas
cuya matriz de valores numéricos viene dada en la tabla siguiente:
Creamos la matriz de decisión:

\begin{Shaded}
\begin{Highlighting}[]
\NormalTok{tabla\_decision }\OtherTok{\textless{}{-}} \FunctionTok{matrix}\NormalTok{(}\FunctionTok{c}\NormalTok{(}\DecValTok{7}\NormalTok{, }\DecValTok{5}\NormalTok{, }\DecValTok{8}\NormalTok{,   }\CommentTok{\# Publicidad TV}
                           \DecValTok{4}\NormalTok{, }\DecValTok{6}\NormalTok{, }\DecValTok{3}\NormalTok{,   }\CommentTok{\# Publicidad Online}
                           \DecValTok{2}\NormalTok{, }\DecValTok{9}\NormalTok{, }\DecValTok{7}\NormalTok{,   }\CommentTok{\# Eventos Presenciales}
                           \DecValTok{5}\NormalTok{, }\DecValTok{4}\NormalTok{, }\DecValTok{6}\NormalTok{),  }\CommentTok{\# Influencers}
                         \AttributeTok{nrow =} \DecValTok{4}\NormalTok{, }\AttributeTok{byrow =} \ConstantTok{TRUE}\NormalTok{)}

\FunctionTok{rownames}\NormalTok{(tabla\_decision) }\OtherTok{\textless{}{-}} \FunctionTok{c}\NormalTok{(}\StringTok{"Publicidad TV"}\NormalTok{, }\StringTok{"Publicidad Online"}\NormalTok{, }\StringTok{"Eventos Presenciales"}\NormalTok{, }\StringTok{"Influencers"}\NormalTok{)}
\FunctionTok{colnames}\NormalTok{(tabla\_decision) }\OtherTok{\textless{}{-}} \FunctionTok{c}\NormalTok{(}\StringTok{"Mercado Local"}\NormalTok{, }\StringTok{"Mercado Internacional"}\NormalTok{, }\StringTok{"Nuevas Tecnologías"}\NormalTok{)}
\NormalTok{tabla\_decision}
\end{Highlighting}
\end{Shaded}

\begin{verbatim}
##                      Mercado Local Mercado Internacional Nuevas Tecnologías
## Publicidad TV                    7                     5                  8
## Publicidad Online                4                     6                  3
## Eventos Presenciales             2                     9                  7
## Influencers                      5                     4                  6
\end{verbatim}

Donde las estrategias son los tipos de publicidad y los estados de la
naturaleza los mercados considerar: a)Beneficios (favorable)

Criterio de wald: Selecciona la estrategia que tenga el mayor de los
valores mínimos, siendo conservador.

\begin{Shaded}
\begin{Highlighting}[]
\NormalTok{s01\_wald }\OtherTok{=} \FunctionTok{criterio.Wald}\NormalTok{(tabla\_decision,T)}
\NormalTok{s01\_wald}
\end{Highlighting}
\end{Shaded}

\begin{verbatim}
## $criterio
## [1] "Wald"
## 
## $metodo
## [1] "favorable"
## 
## $tablaX
##                      Mercado Local Mercado Internacional Nuevas Tecnologías
## Publicidad TV                    7                     5                  8
## Publicidad Online                4                     6                  3
## Eventos Presenciales             2                     9                  7
## Influencers                      5                     4                  6
## 
## $ValorAlternativas
##        Publicidad TV    Publicidad Online Eventos Presenciales 
##                    5                    3                    2 
##          Influencers 
##                    4 
## 
## $ValorOptimo
## [1] 5
## 
## $AlternativaOptima
## Publicidad TV 
##             1
\end{verbatim}

\begin{Shaded}
\begin{Highlighting}[]
\FunctionTok{names}\NormalTok{(s01\_wald}\SpecialCharTok{$}\NormalTok{AlternativaOptima)}
\end{Highlighting}
\end{Shaded}

\begin{verbatim}
## [1] "Publicidad TV"
\end{verbatim}

Criterio Optimista: Este criterio selecciona la estrategia con el mejor
resultado posible, siendo muy optimista.

\begin{Shaded}
\begin{Highlighting}[]
\NormalTok{s01\_optima}\OtherTok{=}\FunctionTok{criterio.Optimista}\NormalTok{(tabla\_decision,T)}
\NormalTok{s01\_optima}
\end{Highlighting}
\end{Shaded}

\begin{verbatim}
## $criterio
## [1] "Optimista"
## 
## $metodo
## [1] "favorable"
## 
## $tablaX
##                      Mercado Local Mercado Internacional Nuevas Tecnologías
## Publicidad TV                    7                     5                  8
## Publicidad Online                4                     6                  3
## Eventos Presenciales             2                     9                  7
## Influencers                      5                     4                  6
## 
## $ValorAlternativas
##        Publicidad TV    Publicidad Online Eventos Presenciales 
##                    8                    6                    9 
##          Influencers 
##                    6 
## 
## $ValorOptimo
## [1] 9
## 
## $AlternativaOptima
## Eventos Presenciales 
##                    3
\end{verbatim}

\begin{Shaded}
\begin{Highlighting}[]
\FunctionTok{names}\NormalTok{(s01\_optima}\SpecialCharTok{$}\NormalTok{AlternativaOptima)}
\end{Highlighting}
\end{Shaded}

\begin{verbatim}
## [1] "Eventos Presenciales"
\end{verbatim}

Criterio Hurwicz:ombina el optimismo y el pesimismo mediante un
coeficiente alfa (en este caso, 0.4, lo que significa que se considera
más pesimismo).

\begin{Shaded}
\begin{Highlighting}[]
\NormalTok{s01\_hurwitz}\OtherTok{=}\FunctionTok{criterio.Hurwicz}\NormalTok{(tabla\_decision,}\AttributeTok{alfa=}\FloatTok{0.4}\NormalTok{,T)}
\NormalTok{s01\_hurwitz}
\end{Highlighting}
\end{Shaded}

\begin{verbatim}
## $criterio
## [1] "Hurwicz"
## 
## $alfa
## [1] 0.4
## 
## $metodo
## [1] "favorable"
## 
## $tablaX
##                      Mercado Local Mercado Internacional Nuevas Tecnologías
## Publicidad TV                    7                     5                  8
## Publicidad Online                4                     6                  3
## Eventos Presenciales             2                     9                  7
## Influencers                      5                     4                  6
## 
## $ValorAlternativas
##        Publicidad TV    Publicidad Online Eventos Presenciales 
##                  6.2                  4.2                  4.8 
##          Influencers 
##                  4.8 
## 
## $ValorOptimo
## [1] 6.2
## 
## $AlternativaOptima
## Publicidad TV 
##             1
\end{verbatim}

\begin{Shaded}
\begin{Highlighting}[]
\FunctionTok{names}\NormalTok{(s01\_hurwitz}\SpecialCharTok{$}\NormalTok{AlternativaOptima)}
\end{Highlighting}
\end{Shaded}

\begin{verbatim}
## [1] "Publicidad TV"
\end{verbatim}

Criterio Savage: Minimiza el arrepentimiento, es decir, la diferencia
entre lo que se obtuvo y lo que podría haberse obtenido en el mejor
escenario.

\begin{Shaded}
\begin{Highlighting}[]
\NormalTok{s01\_savage}\OtherTok{=}\FunctionTok{criterio.Savage}\NormalTok{(tabla\_decision,T)}
\NormalTok{s01\_savage}
\end{Highlighting}
\end{Shaded}

\begin{verbatim}
## $criterio
## [1] "Savage"
## 
## $metodo
## [1] "favorable"
## 
## $tablaX
##                      Mercado Local Mercado Internacional Nuevas Tecnologías
## Publicidad TV                    7                     5                  8
## Publicidad Online                4                     6                  3
## Eventos Presenciales             2                     9                  7
## Influencers                      5                     4                  6
## 
## $Mejores
##         Mercado Local Mercado Internacional    Nuevas Tecnologías 
##                     7                     9                     8 
## 
## $Pesos
##                      Mercado Local Mercado Internacional Nuevas Tecnologías
## Publicidad TV                    0                     4                  0
## Publicidad Online                3                     3                  5
## Eventos Presenciales             5                     0                  1
## Influencers                      2                     5                  2
## 
## $ValorAlternativas
##        Publicidad TV    Publicidad Online Eventos Presenciales 
##                    4                    5                    5 
##          Influencers 
##                    5 
## 
## $ValorOptimo
## [1] 4
## 
## $AlternativaOptima
## Publicidad TV 
##             1
\end{verbatim}

\begin{Shaded}
\begin{Highlighting}[]
\FunctionTok{names}\NormalTok{(s01\_savage}\SpecialCharTok{$}\NormalTok{AlternativaOptima)}
\end{Highlighting}
\end{Shaded}

\begin{verbatim}
## [1] "Publicidad TV"
\end{verbatim}

Criterio Laplace:Considera que todos los estados de la naturaleza son
igualmente probables.

\begin{Shaded}
\begin{Highlighting}[]
\NormalTok{s01\_laplace}\OtherTok{=}\FunctionTok{criterio.Laplace}\NormalTok{(tabla\_decision,T)}
\FunctionTok{names}\NormalTok{(s01\_laplace}\SpecialCharTok{$}\NormalTok{AlternativaOptima)}
\end{Highlighting}
\end{Shaded}

\begin{verbatim}
## [1] "Publicidad TV"
\end{verbatim}

\begin{Shaded}
\begin{Highlighting}[]
\NormalTok{s01\_laplace}\SpecialCharTok{$}\NormalTok{ValorAlternativas}
\end{Highlighting}
\end{Shaded}

\begin{verbatim}
##        Publicidad TV    Publicidad Online Eventos Presenciales 
##             6.666667             4.333333             6.000000 
##          Influencers 
##             5.000000
\end{verbatim}

Criterio Punto Ideal: Compara las alternativas con un ``punto ideal'' en
el que se maximiza todo.

\begin{Shaded}
\begin{Highlighting}[]
\NormalTok{s01\_pid}\OtherTok{=}\FunctionTok{criterio.PuntoIdeal}\NormalTok{(tabla\_decision,T)}
\NormalTok{s01\_pid}\SpecialCharTok{$}\NormalTok{AlternativaOptima}
\end{Highlighting}
\end{Shaded}

\begin{verbatim}
## Publicidad TV 
##             1
\end{verbatim}

\begin{Shaded}
\begin{Highlighting}[]
\NormalTok{s01\_pid}\SpecialCharTok{$}\NormalTok{ValorAlternativas}
\end{Highlighting}
\end{Shaded}

\begin{verbatim}
##        Publicidad TV    Publicidad Online Eventos Presenciales 
##             4.000000             6.557439             5.099020 
##          Influencers 
##             5.744563
\end{verbatim}

Todos los criterios:

\begin{Shaded}
\begin{Highlighting}[]
\NormalTok{s01\_todos}\OtherTok{=}\FunctionTok{criterio.Todos}\NormalTok{(tabla\_decision,}\AttributeTok{alfa=}\FloatTok{0.5}\NormalTok{,T)}
\NormalTok{s01\_todos}
\end{Highlighting}
\end{Shaded}

\begin{verbatim}
##                      Mercado Local Mercado Internacional Nuevas Tecnologías
## Publicidad TV                    7                     5                  8
## Publicidad Online                4                     6                  3
## Eventos Presenciales             2                     9                  7
## Influencers                      5                     4                  6
## iAlt.Opt (fav.)                 --                    --                 --
##                               Wald            Optimista       Hurwicz
## Publicidad TV                    5                    8           6.5
## Publicidad Online                3                    6           4.5
## Eventos Presenciales             2                    9           5.5
## Influencers                      4                    6           5.0
## iAlt.Opt (fav.)      Publicidad TV Eventos Presenciales Publicidad TV
##                             Savage       Laplace   Punto Ideal
## Publicidad TV                    4         6.667         4.000
## Publicidad Online                5         4.333         6.557
## Eventos Presenciales             5         6.000         5.099
## Influencers                      5         5.000         5.745
## iAlt.Opt (fav.)      Publicidad TV Publicidad TV Publicidad TV
\end{verbatim}

\begin{Shaded}
\begin{Highlighting}[]
\FunctionTok{dibuja.criterio.Hurwicz}\NormalTok{(tabla\_decision,T)}
\end{Highlighting}
\end{Shaded}

\includegraphics{Resolucion_files/figure-latex/unnamed-chunk-49-1.pdf}

\begin{Shaded}
\begin{Highlighting}[]
\FunctionTok{dibuja.criterio.Hurwicz\_Intervalos}\NormalTok{(tabla\_decision,T)}
\end{Highlighting}
\end{Shaded}

\includegraphics{Resolucion_files/figure-latex/unnamed-chunk-49-2.pdf}

\begin{verbatim}
## $AltOptimas
## [1] 1 3
## 
## $PuntosDeCorte
## [1] 0.75
## 
## $IntervalosAlfa
##      Intervalo      Alternativa
## [1,] "( 0 , 0.75 )" "1"        
## [2,] "( 0.75 , 1 )" "3"
\end{verbatim}

En todos los casos la mejor estrategia es la Publicidad en TV, excepto
en el criterio optimista que la estrategia son los Eventos Presenciales.

b)Costos (desfavorable) Criterio de wald

\begin{Shaded}
\begin{Highlighting}[]
\NormalTok{s01\_wald }\OtherTok{=} \FunctionTok{criterio.Wald}\NormalTok{(tabla\_decision,F)}
\FunctionTok{names}\NormalTok{(s01\_wald}\SpecialCharTok{$}\NormalTok{AlternativaOptima)}
\end{Highlighting}
\end{Shaded}

\begin{verbatim}
## [1] "Publicidad Online" "Influencers"
\end{verbatim}

Criterio Optimista

\begin{Shaded}
\begin{Highlighting}[]
\NormalTok{s01\_optima}\OtherTok{=}\FunctionTok{criterio.Optimista}\NormalTok{(tabla\_decision,F)}
\FunctionTok{names}\NormalTok{(s01\_optima}\SpecialCharTok{$}\NormalTok{AlternativaOptima)}
\end{Highlighting}
\end{Shaded}

\begin{verbatim}
## [1] "Eventos Presenciales"
\end{verbatim}

Criterio Hurwicz

\begin{Shaded}
\begin{Highlighting}[]
\NormalTok{s01\_hurwicz}\OtherTok{=}\FunctionTok{criterio.Hurwicz}\NormalTok{(tabla\_decision,}\AttributeTok{alfa=}\FloatTok{0.4}\NormalTok{,F)}
\FunctionTok{names}\NormalTok{(s01\_hurwitz}\SpecialCharTok{$}\NormalTok{AlternativaOptima)}
\end{Highlighting}
\end{Shaded}

\begin{verbatim}
## [1] "Publicidad TV"
\end{verbatim}

Criterio Savage

\begin{Shaded}
\begin{Highlighting}[]
\NormalTok{s01\_savage}\OtherTok{=}\FunctionTok{criterio.Savage}\NormalTok{(tabla\_decision,F)}
\FunctionTok{names}\NormalTok{(s01\_savage}\SpecialCharTok{$}\NormalTok{AlternativaOptima)}
\end{Highlighting}
\end{Shaded}

\begin{verbatim}
## [1] "Publicidad Online"
\end{verbatim}

Criterio Laplace

\begin{Shaded}
\begin{Highlighting}[]
\NormalTok{s01\_laplace}\OtherTok{=}\FunctionTok{criterio.Laplace}\NormalTok{(tabla\_decision,F)}
\FunctionTok{names}\NormalTok{(s01\_laplace}\SpecialCharTok{$}\NormalTok{AlternativaOptima)}
\end{Highlighting}
\end{Shaded}

\begin{verbatim}
## [1] "Publicidad Online"
\end{verbatim}

\begin{Shaded}
\begin{Highlighting}[]
\NormalTok{s01\_laplace}\SpecialCharTok{$}\NormalTok{ValorAlternativas}
\end{Highlighting}
\end{Shaded}

\begin{verbatim}
##        Publicidad TV    Publicidad Online Eventos Presenciales 
##             6.666667             4.333333             6.000000 
##          Influencers 
##             5.000000
\end{verbatim}

Criterio Punto Ideal

\begin{Shaded}
\begin{Highlighting}[]
\NormalTok{s01\_pid}\OtherTok{=}\FunctionTok{criterio.PuntoIdeal}\NormalTok{(tabla\_decision,F)}
\NormalTok{s01\_pid}\SpecialCharTok{$}\NormalTok{AlternativaOptima}
\end{Highlighting}
\end{Shaded}

\begin{verbatim}
## Publicidad Online 
##                 2
\end{verbatim}

\begin{Shaded}
\begin{Highlighting}[]
\NormalTok{s01\_pid}\SpecialCharTok{$}\NormalTok{ValorAlternativas}
\end{Highlighting}
\end{Shaded}

\begin{verbatim}
##        Publicidad TV    Publicidad Online Eventos Presenciales 
##             7.141428             2.828427             6.403124 
##          Influencers 
##             4.242641
\end{verbatim}

Todos los criterios:

\begin{Shaded}
\begin{Highlighting}[]
\NormalTok{s01\_todos}\OtherTok{=}\FunctionTok{criterio.Todos}\NormalTok{(tabla\_decision,}\AttributeTok{alfa=}\FloatTok{0.5}\NormalTok{,F)}
\NormalTok{s01\_todos}
\end{Highlighting}
\end{Shaded}

\begin{verbatim}
##                      Mercado Local Mercado Internacional Nuevas Tecnologías
## Publicidad TV                    7                     5                  8
## Publicidad Online                4                     6                  3
## Eventos Presenciales             2                     9                  7
## Influencers                      5                     4                  6
## iAlt.Opt (Desfav.)              --                    --                 --
##                                               Wald            Optimista
## Publicidad TV                                    8                    5
## Publicidad Online                                6                    3
## Eventos Presenciales                             9                    2
## Influencers                                      6                    4
## iAlt.Opt (Desfav.)   Publicidad Online,Influencers Eventos Presenciales
##                                Hurwicz            Savage           Laplace
## Publicidad TV                      6.5                 5             6.667
## Publicidad Online                  4.5                 2             4.333
## Eventos Presenciales               5.5                 5             6.000
## Influencers                        5.0                 3             5.000
## iAlt.Opt (Desfav.)   Publicidad Online Publicidad Online Publicidad Online
##                            Punto Ideal
## Publicidad TV                    7.141
## Publicidad Online                2.828
## Eventos Presenciales             6.403
## Influencers                      4.243
## iAlt.Opt (Desfav.)   Publicidad Online
\end{verbatim}

\begin{Shaded}
\begin{Highlighting}[]
\FunctionTok{dibuja.criterio.Hurwicz}\NormalTok{(tabla\_decision,F)}
\end{Highlighting}
\end{Shaded}

\includegraphics{Resolucion_files/figure-latex/unnamed-chunk-56-1.pdf}

\begin{Shaded}
\begin{Highlighting}[]
\FunctionTok{dibuja.criterio.Hurwicz\_Intervalos}\NormalTok{(tabla\_decision,F)}
\end{Highlighting}
\end{Shaded}

\includegraphics{Resolucion_files/figure-latex/unnamed-chunk-56-2.pdf}

\begin{verbatim}
## $AltOptimas
## [1] 2 3
## 
## $PuntosDeCorte
## [1] 0.75
## 
## $IntervalosAlfa
##      Intervalo      Alternativa
## [1,] "( 0 , 0.75 )" "2"        
## [2,] "( 0.75 , 1 )" "3"
\end{verbatim}

En 5 de los criterios la mejor opción es la Publicidad Online para
minimizar los costos. El criterio de Wald tiene dos soluciones, la
anterior y la estrategia de Influencers. Por otro lado, el criterio
optimista repite con Eventos Presenciales.

\subsection{Problema 2}\label{problema-2-3}

Dos amigos, Claudia y Mario, están planeando abrir una pequeña
cafetería. Han ahorrado 1500 euros cada uno y quieren decidir qué
enfoque darle al negocio. Existen tres opciones según el tipo de menú
que ofrezcan: Un menú básico, un menú saludable o un menú gourmet.
Dependiendo de la situación económica y las tendencias alimentarias, la
demanda puede aumentar o disminuir, lo que afectará sus ingresos.

Si eligen un menú básico, tendrán un flujo constante de clientes, pero
los márgenes de ganancia serán bajos, con una ganancia o pérdida del
5\%.

Si eligen un menú saludable, pueden atraer a un nicho de clientes en
crecimiento, pero la inversión en ingredientes será mayor, y podrían
obtener una ganancia del 12\% si la tendencia continúa, o perder un 8\%
si la tendencia no se mantiene.

Si optan por un menú gourmet, el riesgo es mayor porque depende de la
clientela de alto poder adquisitivo, lo que les permitirá obtener una
ganancia del 20\% si la economía es favorable, pero podrían perder un
15\% si la economía se desacelera. Claudia es más conservadora y
prefiere minimizar los riesgos, mientras que Mario está dispuesto a
asumir más riesgos confiando en que la economía mejorará.

¿Qué tipo de menú elegiría cada uno de ellos según sus actitudes hacia
el riesgo? Estado e1 (Demanda favorable)

\begin{Shaded}
\begin{Highlighting}[]
\NormalTok{n11 }\OtherTok{=} \DecValTok{1500} \SpecialCharTok{*} \FloatTok{0.05}     \CommentTok{\# Menú básico}
\NormalTok{n21 }\OtherTok{=} \DecValTok{1500} \SpecialCharTok{*} \FloatTok{0.12}     \CommentTok{\# Menú saludable}
\NormalTok{n31 }\OtherTok{=} \DecValTok{1500} \SpecialCharTok{*} \FloatTok{0.20}     \CommentTok{\# Menú gourmet}
\end{Highlighting}
\end{Shaded}

Estado e2 (Demanda desfavorable)

\begin{Shaded}
\begin{Highlighting}[]
\NormalTok{n12 }\OtherTok{=} \SpecialCharTok{{-}}\DecValTok{1500} \SpecialCharTok{*} \FloatTok{0.05}    \CommentTok{\# Menú básico}
\NormalTok{n22 }\OtherTok{=} \SpecialCharTok{{-}}\DecValTok{1500} \SpecialCharTok{*} \FloatTok{0.08}    \CommentTok{\# Menú saludable}
\NormalTok{n32 }\OtherTok{=} \SpecialCharTok{{-}}\DecValTok{1500} \SpecialCharTok{*} \FloatTok{0.15}    \CommentTok{\# Menú gourmet}
\end{Highlighting}
\end{Shaded}

Crear la tabla de decisión

\begin{Shaded}
\begin{Highlighting}[]
\NormalTok{tb\_decision }\OtherTok{=} \FunctionTok{crea.tablaX}\NormalTok{(}\FunctionTok{c}\NormalTok{(n11, n12,}
\NormalTok{                            n21, n22,}
\NormalTok{                            n31, n32), }\DecValTok{3}\NormalTok{, }\DecValTok{2}\NormalTok{)}
\end{Highlighting}
\end{Shaded}

Aplicar los criterios de decisión

\begin{Shaded}
\begin{Highlighting}[]
\NormalTok{res\_decision }\OtherTok{=} \FunctionTok{criterio.Todos}\NormalTok{(tb\_decision, }\AttributeTok{alfa =} \FloatTok{0.5}\NormalTok{, }\AttributeTok{favorable =} \ConstantTok{TRUE}\NormalTok{)}
\end{Highlighting}
\end{Shaded}

Mostrar los resultados en formato tabla

\begin{Shaded}
\begin{Highlighting}[]
\NormalTok{knitr}\SpecialCharTok{::}\FunctionTok{kable}\NormalTok{(res\_decision)}
\end{Highlighting}
\end{Shaded}

\begin{longtable}[]{@{}
  >{\raggedright\arraybackslash}p{(\columnwidth - 16\tabcolsep) * \real{0.2133}}
  >{\raggedright\arraybackslash}p{(\columnwidth - 16\tabcolsep) * \real{0.0533}}
  >{\raggedright\arraybackslash}p{(\columnwidth - 16\tabcolsep) * \real{0.0667}}
  >{\raggedright\arraybackslash}p{(\columnwidth - 16\tabcolsep) * \real{0.0667}}
  >{\raggedright\arraybackslash}p{(\columnwidth - 16\tabcolsep) * \real{0.1333}}
  >{\raggedright\arraybackslash}p{(\columnwidth - 16\tabcolsep) * \real{0.1067}}
  >{\raggedright\arraybackslash}p{(\columnwidth - 16\tabcolsep) * \real{0.0933}}
  >{\raggedright\arraybackslash}p{(\columnwidth - 16\tabcolsep) * \real{0.1067}}
  >{\raggedright\arraybackslash}p{(\columnwidth - 16\tabcolsep) * \real{0.1600}}@{}}
\toprule\noalign{}
\begin{minipage}[b]{\linewidth}\raggedright
\end{minipage} & \begin{minipage}[b]{\linewidth}\raggedright
e1
\end{minipage} & \begin{minipage}[b]{\linewidth}\raggedright
e2
\end{minipage} & \begin{minipage}[b]{\linewidth}\raggedright
Wald
\end{minipage} & \begin{minipage}[b]{\linewidth}\raggedright
Optimista
\end{minipage} & \begin{minipage}[b]{\linewidth}\raggedright
Hurwicz
\end{minipage} & \begin{minipage}[b]{\linewidth}\raggedright
Savage
\end{minipage} & \begin{minipage}[b]{\linewidth}\raggedright
Laplace
\end{minipage} & \begin{minipage}[b]{\linewidth}\raggedright
Punto Ideal
\end{minipage} \\
\midrule\noalign{}
\endhead
\bottomrule\noalign{}
\endlastfoot
d1 & 75 & -75 & -75 & 75 & 0.0 & 225 & 0.0 & 225.0 \\
d2 & 180 & -120 & -120 & 180 & 30.0 & 120 & 30.0 & 128.2 \\
d3 & 300 & -225 & -225 & 300 & 37.5 & 150 & 37.5 & 150.0 \\
iAlt.Opt (fav.) & -- & -- & d1 & d3 & d3 & d2 & d3 & d2 \\
\end{longtable}

Claudia es conservadora y busca minimizar las posibles pérdidas.
Aplicando el criterio de Wald o Minimax, probablemente seleccione la
opción del menú básico, que ofrece la menor pérdida en el peor escenario
(-75 euros). Mario es más arriesgado y busca maximizar sus posibles
ganancias. Aplicando el criterio optimista (Maximax), seleccionará el
menú gourmet, que le podría dar la mayor ganancia en el mejor escenario
(300 euros). Claudia optará por un menú básico y Mario por un menú
gourmet.

\end{document}
